\documentclass[11pt]{article}

\usepackage{epsfig}
\usepackage{amsfonts}
\usepackage{amssymb}
\usepackage{amstext}
\usepackage{amscd}
\usepackage{amsmath}
\usepackage{xspace}
\usepackage{theorem}
\usepackage{float}
\usepackage[table]{xcolor}
\usepackage{color}
\usepackage{pgfplots}

\definecolor{stainlessSteel}{cmyk}{0,0,0.02,0.12}
%\usepackage{layout}% if you want to see the layout parameters
                     % and now use \layout command in the body

% This is the stuff for normal spacing
\makeatletter
 \setlength{\textwidth}{6.5in}
 \setlength{\oddsidemargin}{0in}
 \setlength{\evensidemargin}{0in}
 \setlength{\topmargin}{0.25in}
 \setlength{\textheight}{8.25in}
 \setlength{\headheight}{0pt}
 \setlength{\headsep}{0pt}
 \setlength{\marginparwidth}{59pt}

 \setlength{\parindent}{0pt}
 \setlength{\parskip}{5pt plus 1pt}
 \setlength{\theorempreskipamount}{5pt plus 1pt}
 \setlength{\theorempostskipamount}{0pt}
 \setlength{\abovedisplayskip}{8pt plus 3pt minus 6pt}
 \setlength{\intextsep}{15pt plus 3pt minus 6pt}

 \renewcommand{\section}{\@startsection{section}{1}{0mm}%
                                   {2ex plus -1ex minus -.2ex}%
                                   {1.3ex plus .2ex}%
                                   {\normalfont\Large\bfseries}}%
 \renewcommand{\subsection}{\@startsection{subsection}{2}{0mm}%
                                     {1ex plus -1ex minus -.2ex}%
                                     {1ex plus .2ex}%
                                     {\normalfont\large\bfseries}}%
 \renewcommand{\subsubsection}{\@startsection{subsubsection}{3}{0mm}%
                                     {1ex plus -1ex minus -.2ex}%
                                     {1ex plus .2ex}%
                                     {\normalfont\normalsize\bfseries}}
 \renewcommand\paragraph{\@startsection{paragraph}{4}{0mm}%
                                    {1ex \@plus1ex \@minus.2ex}%
                                    {-1em}%
                                    {\normalfont\normalsize\bfseries}}
 \renewcommand\subparagraph{\@startsection{subparagraph}{5}{\parindent}%
                                       {2.0ex \@plus1ex \@minus .2ex}%
                                       {-1em}%
                                      {\normalfont\normalsize\bfseries}}
\makeatother

\newcounter{thelecture}

\newenvironment{proof}{{\bf Proof:  }}{\hfill\rule{2mm}{2mm}}
\newenvironment{proofof}[1]{{\bf Proof of #1:  }}{\hfill\rule{2mm}{2mm}}
\newenvironment{proofofnobox}[1]{{\bf#1:  }}{}
\newenvironment{example}{{\bf Example:  }}{\hfill\rule{0mm}{0mm}} % change 2mm 2mm for square

%\renewcommand{\theequation}{\thesection.\arabic{equation}}
%\renewcommand{\thefigure}{\thesection.\arabic{figure}}

\newtheorem{fact}{Fact}
\newtheorem{lemma}[fact]{Lemma}
\newtheorem{theorem}[fact]{Theorem}
\newtheorem{definition}[fact]{Definition}
\newtheorem{corollary}[fact]{Corollary}
\newtheorem{proposition}[fact]{Proposition}
\newtheorem{claim}[fact]{Claim}
\newtheorem{exercise}[fact]{Exercise}

% math notation
\newcommand{\R}{\ensuremath{\mathbb R}}
\newcommand{\Z}{\ensuremath{\mathbb Z}}
\newcommand{\N}{\ensuremath{\mathbb N}}
\newcommand{\B}{\ensuremath{\mathbb B}}
\newcommand{\F}{\ensuremath{\mathcal F}}
\newcommand{\SymGrp}{\ensuremath{\mathfrak S}}
\newcommand{\prob}[1]{\ensuremath{\text{{\bf Pr}$\left[#1\right]$}}}
\newcommand{\expct}[1]{\ensuremath{\text{{\bf E}$\left[#1\right]$}}}
\newcommand{\size}[1]{\ensuremath{\left|#1\right|}}
\newcommand{\ceil}[1]{\ensuremath{\left\lceil#1\right\rceil}}
\newcommand{\floor}[1]{\ensuremath{\left\lfloor#1\right\rfloor}}
\newcommand{\ang}[1]{\ensuremath{\langle{#1}\rangle}}
\newcommand{\poly}{\operatorname{poly}}
\newcommand{\polylog}{\operatorname{polylog}}

% anupam's abbreviations
\newcommand{\e}{\epsilon}
\newcommand{\half}{\ensuremath{\frac{1}{2}}}
\newcommand{\junk}[1]{}
\newcommand{\sse}{\subseteq}
\newcommand{\union}{\cup}
\newcommand{\meet}{\wedge}
\newcommand{\dist}[1]{\|{#1}\|_{\text{dist}}}
\newcommand{\hooklongrightarrow}{\lhook\joinrel\longrightarrow}
\newcommand{\embeds}[1]{\;\lhook\joinrel\xrightarrow{#1}\;}
\newcommand{\mnote}[1]{\normalmarginpar \marginpar{\tiny #1}}

%%%%%%%%%%%%%%%%%%%%%%%%%%%%%%%%%%%%%%%%%%%%%%%%%%%%%%%%%%%%%%%%%%%%%%%%%%%
% Document begins here %%%%%%%%%%%%%%%%%%%%%%%%%%%%%%%%%%%%%%%%%%%%%%%%%%%%
%%%%%%%%%%%%%%%%%%%%%%%%%%%%%%%%%%%%%%%%%%%%%%%%%%%%%%%%%%%%%%%%%%%%%%%%%%%

\newcommand{\hwheadings}[3]{
{\bf Calculus 200 -  Fall 2015} \hfill {{\bf Exam 2 study guide}}\\
{{\bf } #2} \hfill {{\bf Due:} #3} \\
\rule[0.1in]{\textwidth}{0.025in}
%\thispagestyle{empty}
}
\begin{document}
\hwheadings{4}{13.2-14.8}{Nov 11, 2015}

\begin{enumerate}

\item 13.2 \textbf{Calculus with vector functions}
\\
$T = \frac{r'}{|r'|} \; \; \; \cod \theta= \frac{r' s'}{|r'||s'|}$
\\





\item 14.1  \textbf{Functions of several variables}




\item 14.2 \textbf{Limits and Continuity}
\\
\begin{center}
\textit{definition:} Suppose $f(x,y)$ is a function. We say that
\\
$\lim \limits_{(x,y) \to (a,b)} f(x,y)=L$
\\
if for ever $ \epsilon > 0$ there is a $\delta > 0$ so that whenever $0< \sqrt{(x-a)^2+(y-b)^2}<\delta, \; \; |f(x,y)-L|< \epsilon$.
\end{center}

\\
\begin{center}
\textit{theorem:}
\\
If $|f(x,y)=L| \le g(x,y) \forall (x,y)$ in a circle centered around $(a,b)$ and 
\\
$\lim \limits_{(x,y) \to (a,b)} g(x,y)=0$
\\
then $\lim \limits_{(x,y) \to (a,b)} f(x,y)=L$


\end{center}
\\
\begin{proof}
Let $\epsilon >0$, since 
$\lim \limits_{(x,y) \to (a,b)} g(x,y)=0$
$\exists \delta >0 : |g(x,y)-0|<\epsilon$ whenever
$\sqrt{(x-a)^2+(y-b)^2}< \delta$.
\\
But we also have $|f(x,y)-L| \le g(x,y) , \epsilon$
\\
$|(x,y)-L|< \epsilon$ when $\sqrt{(x-a)^2+(y-b)^2}< \delta$
\\
To use the theorem, find $g(x,y) \ge |f(x,y)-L|$ who's limit is zero.
\\
\end{proof}

\begin{center}
\textit{Example:}
Find and prove the $\lim \limits_{(x,y) \to (0,0)} \frac{x^2y}{x^2+y^2}$ or disprove if the limit DNE.
\\
let $x=0$ and $y \to 0 \; \; \; f(0,y)=\frac{0}{y^2}=0 $ therefore $f \to 0$ as $y \to 0$
\\
Let $x \to 0$ and $y=0 \; \; \; f(x,0)=\frac{0}{x^2}=0$ therefore $f \to 0$ as $x \to 0$
\\
Let $x=y \; \; \; f(x,x)= \frac{x^3}{2x^2}=\frac{x}{2} \; \; \; f \to 0$ as $x \to 0$
\\
$|f(x,y)-L|=|f(x,y)-0|=|f(x,y)|=|\frac{x^2y}{x^2+y^2}|=\frac{x^2}{x^2+y^2}|y| \le |y|= g(x,y)$
\\
$\lim \limits_{(y) \to (0)} |y|=0$
\\
Therefore we proved that $\lim \limits_{(x,y) \to (0,0)} \frac{x^2y}{x^2+y^2}=0$


\end{center}
\\
\textit{definition:} $f(x,y)$ is continuous at $(a,b)$ if $\lim \limits_{(x,y) \to (a,b)} f(x,y)=(a,b)$
\\
-all polynomials are continuous
\\
-all rational functions are continuous on there domain
\\
-ln is continuous on its domain










\item 14.3 Partial derivatives
\\
$\frac{\partial f}{\partial x}=f_x= \lim \limits_{h \to 0} \frac{f(x+h,y)-f(x,y)}{h}$
\\
$\frac{\partial^2 f}{\partial^2 x}=f_{xx}$
\\
$f_{xy}=f_{yx}$ if the function is continuous. 
\\
\begin{center}
\textit{Example:}
In a chemical reaction, the temperature $T$, entropy $S$, Gibbs free enger $G$, and enthalpy $H$ are related by $G = H - TS$.  Show that $\frac{\partial(G/T)}{\partial T} =  - \frac{H}{T^2}$.
    \\
   
    \\
    $G=H-TS$
    \\
    $(G/T)=\frac{H}{T}-S$
    \\
    $\frac{\partial(G/T)}{\partial T}=-\frac{H}{T^2}$


\end{center}





\item 14.4 The Chain Rule

\textit{Taylor Series:} 
$z=f(a,b)+f_x(a,b)(x-a)+f_y(a,b)(y-b)$
\\
\textit{Mean Value theorem:}
There is at least one point $c \in (a,b)$ where the slope of the tangent line at $c$ equals the slope of the secant line from $(a,f(a))$ to $(b,f(b))$
\\
$f'(c)=\frac{f(b)-f(a)}{b-a} \rightarrow f(b)-f(a)=f'(c)[b-a] \rightarrow \Delta y = f'(c) \Delta x \rightaarow f'(c)=\frac{\delta y}{\delta x}$
\\
\textit{tangent line:} $n_1(x-a)+n_2(y-b)+n_3(z-c)=0$ at the point $(a,b,f(a,b))$ with the normal vector $\langle n_1, n_2, n_3 \rangle$
\\
\begin{center}
\textit{theorem:} 
suppose that $z=f(x,y)$, $f$ is differentiable, $x=g(t)$ and $y=h(t)$ Assuming that the relevant derivatives exist,
\\
$\frac{dz}{dt}=\frac{\partial z}{\partial x}\frac{dx}{dt}+\frac{\partial z}{\partial y} \frac{dy}{dt}$
\\
\begin{proof}
If $f$ is differentiable, then
\\
$\Delta z= f_x (x_0,y_0) \Delta x+ f_y (x_0,y_0) \Delta y + \epsilon_1 \Deta x + \epsilon_2 \Delta y$
\\
Where $\epsilon_1$ and $\epsilon_2$ approuch $0$ as $(x,y)$ approaches $(x_0,y_0)$. Then
\\
$\frac{\Delta z}{\Delta t}=f_x\frac{\Delta x}{\Delta t}+ f_y \frac{\Delta y}{\Delta t}+\epsilon_1 \frac{\Delta x}{\Delta t}+ \epsilon_2 \frac{\Delta y}{\Delta t}$.
\\
As $\Delta t$ approaches $0$, $(x,y)$ approaches $(x_0,y_0)$ and so
\\
$\lim \limits_{\Delta t \to 0} \frac{\Delta z}{\Delta t}=\frac{dz}{dt}$
\\
$\lim \limits_{\Delta t \to 0} \epsilon_1 \frac{\Delta x}{\Delta t}= 0 \frac{dx}{dt}$
\\
$\lim \limits_{\Delta t \to 0} \epsilon_2 \frac{\Delta y}{\Delta t}= 0 \frac{dy}{dt}$
\\
And so taking the limit as $\Delta t$ goes to $0$ gives
\\
$\frac{dz}{dt}=f_x\frac{dx}{dt}+f_y \frac{dy}{dt}$


\end{proof}


\end{center}


\textit{Example:}
\\
 Suppose $f(x,y) = x^2 e^y$, $x(t) = t^2 - 1$, and $y(t) = \sin t$, find the derivative of $g(t) = f(x(t), y(t))$.
    \\
    \textit{Answer:}
\\
$g(t)'=\frac{\partial g(t)}{\partial x} \frac{dx}{dt} + \frac{\partial g(t)}{\partial y} \frac{dy}{dt}$
\\
$\frac{\partial g(t)}{\partial x}= 2xe^y$
\\
$\frac{\partial g(t)}{\partial y}= x^2e^y$
\\
$\frac{dx}{dt}=2t$
\\
$\frac{dy}{dt}= \cos (t)$
\\
Therefore
\\
$g(t)'=(2xe^y)(2t)+(x^2e^y)(\cos (t)= 4txe^y+x^2e^y \cos (t)$



\\


\textit{multi-points}
\\
$x=x(s,t) \; \; \; y=y(s,t) \; \; \; z=f(x,y)$
\\
$\frac{\partial z}{\partial s}= \frac{\partial z}{\partial x}\frac{\partial x}{\partial s}+ \frac{\partial z}{\partial y} \frac{\partial y}{\partial y}{\partial s}$
\\
$\frac{\partial z}{\partial t}= \frac{\partial z}{\partial x}\frac{\partial x}{\partial t}+ \frac{\partial z}{\partial y} \frac{\partial y}{\partial y}{\partial t}$

\\

\textit{Example:}
\\
Suppose $f(x,y) = e^{xy}$, $x(u,v) = 3u \sin v$ and $y(u,v) = 4v^2u$.  For $g(u,v) = f(x(u,v), y(u,v))$, find the partial derivatives $\frac{\partial g}{\partial u}$ and $\frac{\partial g}{\partial v}$.
    \\
    \textbf{Answer:}
    \\
    $\frac{\partial g}{\partial u}=\frac{\partial g}{\partial x} \frac{\partial x}{\partial u}+ \frac{\partial g}{\partial y} \frac{\partial y}{\partial u}$
    \\
    $\frac{\partial g}{\partial x}=ye^{xy}$
    \\
    $\frac{\partial x}{\partial u}=3 \sin (v)$
    \\
    $\frac{\partial g}{\partial y}=xe^{xy}$
    \\
    $\frac{\partial y}{\partial u}=4v^2$
    \\
    $\frac{\partial g}{\partial u}=(ye^{xy})(3 \sin (v))+(=xe^{xy})(4v^2)$
    
    \\
    \\
    $\frac{\partial g}{\partial v}=\frac{\partial g}{\partial x}\frac{\partial x}{\partial v}+\frac{\partial g}{\partial y}\frac{\partial y}{\partial v}$
    \\
    $\frac{\partial g}{\partial x}=ye^{xy}$
    \\
    $\frac{\partial x}{\partial v}=3u \cos(v)$
    \\
    $\frac{\partial g}{\partial y}=xe^{xy}$
    \\
    $\frac{\partial y}{\partial v}=8vu$
    \\
    $\frac{\partial g}{\partial v}=(ye^{xy})(3u \cos(v))+ (xe^{xy}) (8vu) $



\item 14.5 \textbf{Directional Derivatives}
\\
\begin{center}
\textit{Definition:}
Directional derivative of $z=f(x,y)$ at point $(a,b)$ in the direction at $u$ (unit vector) $\langle u_1, u_2 \rangle$
\\
$D_u f (a,b)= f_x u_1 + f_y u_2 = \langle f_x, f_y \rangle \cdot \langle u_1, u_2 \rangle = \nabla f \cdot u= |\nabla f||u| \cos \theta= |\nabla f| \cos \theta$
\end{center}
\\
\textit{Example:}
\\
 Find the directional derivative $D_{{\bf u}}f(1, -1)$ when $f(x,y) = x^2 + y^2$ for ${\bf u}$ in the direction $\langle -3, 4 \rangle$. 
    \\
    \textit{Answer:}
    \\
    Make the direction vector into a directional unit vector by setting ${\bf u}=\frac{\bf u}{||u||}$
    \\
    $\langle \frac{-3}{5}, \frac{4}{5} \rangle$
    \\
    
    $D_{{\bf u}}f(a, b)=f_x u_1+f_y u_2$ where $u_1,u_2$ at the components of the direction unit vector. Therefore:
    \\
    $D_{{\bf u}}f(1, -1)=f_x(\frac{-3}{5})+f_y(\frac{4}{5})$
    \\
    $f_x=2x$
    \\
    $f_y=2y$
    \\
    $D_{{\bf u}}f(1, -1)=2x(\frac{-3}{5})+2y(\frac{4}{5})=\frac{-6}{5}x+\frac{8}{5}y$
    
    
    
    \\
    \\
    \textit{Example:}
    \\
    Find the directional derivative of $f(x,y) = x^3y^4+x^4y^3$ at the point $(1,1)$ in the direction $\theta =\pi/6$.
    \\
    \textit{Answer:}
    \\
    $D_uf(a,b)= \langle f_x, f_y \rangle \cdot \langle u_1,u_2 \rangle$
    \\
    $\nabla f \cdot u= ||f|| \times ||u|| \times \cos \theta$
    \\
    $f_x=3x^2y^4+4x^3y^3$
    \\
    $f_y=4y^3x^3+3y^2x^4$
    \\
    $\nabla f= \langle 3x^2y^4+4x^3y^3, 4y^3x^3+3y^2x^4 \rangle$
    \\
    $\nabla f(1,1)= \langle 3(1)^2(1)^4+4(1)^3(1)^3, 4(1)^3(1)^3+3(1)^2(1)^4 \rangle$
    \\
     $\nabla f(1,1)= \langle 7,7 \rangle$
     \\
     $||\nabla f(1,1)||=\sqrt{98}$
     \\
     $||u||=1$ because $||u||$ is a unit vector.
     \\
     $D_uf(1,1)= \sqrt{98} \times 1 \times \cos{\pi/6}= \sqrt{98} \times \frac{\sqrt{3}}{2}= \frac{\sqrt{294}}{2} $
     
     \\
     
     \\
     
     \begin{proof}
     
   \item Prove that $f$ is a differentiable function of two or three variables then the maximum value of the directional derivative $D_{\bf u}f({\bf x})$ is $| \nabla f({\bf x})|$ and occurs when ${\bf u}$ is in the direction of the gradient vector.
   \\
   \textit{Answer:}
   \\
   We know that $D_uf(a,b)= \langle f_x, f_y \rangle \cdot \langle u_1,u_2 \rangle$ or $D_uf(a,b)= ||\langle f_x, f_y \rangle|| \times ||u|| \times \cos \theta$ We also know that $||u||=1$ because $u$ is the unit vector. Finally, we know that $-1 \ge \cos \theta \le 1$.
   \\
   We can simplify $D_uf(a,b)$ to $D_uf(a,b)= ||\nabla f(x)|| \times \cos \theta$  
   \\
   If we take the max of $D_uf(a,b)$ we will see that the max of $\cos \theta =1$ and therefore the max of $D_uf(a,b)= ||\nabla f(x)||$. 
   \\
   This max occurs when the $\cos \theta =1$ and we know this happens when $\theta = 0$. $\theta$ is the angle between $u$ and the gradient vector, therefore if $u$ is in the direction of the gradient vector $\theta$ will be $0$, the $\cos \theta =1$ and $D_uf(a,b)= ||\nabla f(x)||$. 
     
     
     \end{proof}





\item 14.6 \textbf{Higher order Derivations}
\\
\textit{Critical points:}
$(a,b)$ is a critical point of $f(x,y)$ if $f_x(a,b)=f_y(a,b)=0$ or one or both partial derivatives do not exist.
\\
\textit{Example:}
\\
Find all critical points of $f(x,y)=x^2+xy+y^2+y$
\\
$f_x(x,y)=2x+y \; \; \; f_y(x,y)=x+2y+1$
\\
$2x+y=0 \rightarrow y=-2x$
\\
$x+2y+=0 \rightarrow x+2(-2x)+1=0 \rightarrow x-4x+1=0 \rightarrow -3x=-1$
\\
$x=\frac{1}{3} \; \; \; y=\frac{2}{3}$






\item 14.7 \textbf{Maxima and Minima}
\\
\textit{Definition:}
A local max occurs at $(a,b)$ if $f(a,b) \ge f(x,y) \forall (x,y) \in D ((a,b),r)$
\\
\textit{Definition:}
A local max occurs at $(a,b)$ if $f(a,b) \le f(x,y) \forall (x,y) \in D ((a,b),r)$

\textbf{Every max and min is a critical point but not every critical point is a max or min}
\\
\textit{theorem}
\\
Suppose that the second partial derivatives of $f(x,y)$ are continuous near $(x_0,y_0)$, and $f_x(x_0,y_0)=f_y(x_0,y_0)=0$. We denote by $D$ the discriminant.
\\
$D(x_0,y0)=f_{xx}(x_0,y_0)f_{yy}(x_0,y_0)-f_{xy}(x_0,y_0)^2$.
\begin{center}
if $f_x(x_0,y_0)=0=f_y(x_0,y_0$ and:
\\
$D(x_0,y0)=f_{xx}(x_0,y_0)f_{yy}(x_0,y_0)-f_{xy}(x_0,y_0)^2 > 0$ and $f_{xx}<0$ there is a max at $(x_0,y_0)$
\\
$D(x_0,y0)=f_{xx}(x_0,y_0)f_{yy}(x_0,y_0)-f_{xy}(x_0,y_0)^2 > 0$ and $f_{xx}>0$ there is a min at $(x_0,y_0)$
\\
$D(x_0,y0)=f_{xx}(x_0,y_0)f_{yy}(x_0,y_0)-f_{xy}(x_0,y_0)^2 < 0$ there is a saddle point at $(x_0,y_0)$
\\
$D(x_0,y0)=f_{xx}(x_0,y_0)f_{yy}(x_0,y_0)-f_{xy}(x_0,y_0)^2 = 0$ the test fails.




\end{center}

\\
\textit{Example:}
\\
Find and classify all critical points of $f(x,y)=x^2+xy+y^2+y$
\\
$f_x(x,y)=2x+y \; \; \; f_y(x,y)=x+2y+1$
\\
$2x+y=0 \rightarrow y=-2x$
\\
$x+2y+=0 \rightarrow x+2(-2x)+1=0 \rightarrow x-4x+1=0 \rightarrow -3x=-1$
\\
$x=\frac{1}{3} \; \; \; y=-\frac{2}{3}$
\\
$f_{xx}=2 \; \;f_{yy}=2 \; \; f_{xy}=1 \; \; f_{yx}=1$
\\
$D(x_0,y_0)=f_{xx}(x_0,y_0)f_{yy}(x_0,y_0)-f_{xy}(x_0,y_0)^2=2 \times 2 - 1^2 =3$
\\
$D>0 \; \; f_{xx} >0 \; \; (\frac{1}{3},-\frac{2}{3}$ is a min.
\\

\\
\textit{Example:}





\textit{Extreme Value Theorem:}
If $f(x,y)$ is continuous on a closed bounded region $D \le \mathbb{R} \times \mathbb{R} \approx \mathbb{R}^2$ 
\\
Then $f$ attains its absolute max and min either at the critical points or on thee boundary.

\\
\textit{Example:} Find points on $x-2y+3z=6$ closest to $(0,1,1)$
\\
Distance function:
$d(0,1,1),(x,y,z)=\sqrt{(x)^2+(y-1)^2+(z-1)^2$
\\
$x-2y+3z=6 \rightarrow z=2-\frac{1}{3}x+{2}{3}y$
\\
$d=\sqrt{(x)^2+(y-1)^2+(2-\frac{1}{3}x+{2}{3}y-1)^2$
\\
$f_x=\frac{20}{9}x-\frac{2}{3}-\frac{4}{9}y=0 $
\\
$f_y=\frac{26}{9}y-\frac{2}{3}-\frac{4}{9}x=0$
\\
$(\frac{3}{10}+\frac{5}{63},\frac{25}{63}$
\\
$f_{xx}=\frac{20}{9} >0 \; \; \; f_{yy}=\frac{26}{9} \; \; \; f_{xy}= -\frac{4}{9}$
\\
$D=\frac{20}{9} \frac{26}{9} - (-\frac{4}{9})^2=\frac{505}{81}>0 \rightarrow$ min






\\
 Find the absolute maximum and minimum values of $f(x,y) = x^2 +y^2 - 2x$ on the closed triangle region with vertices $(2,0), (0,2)$, and $(0,-2)$.
    \\
    \textit{Answer:}
    \\
    $f_x=2x-2=0 \; \; \; x=1$
    \\
    $f_y=2y \; \; \; y=0$
    \\
    $(1,0)$
    \\
    E.P.: $(1,0),(2,0),(0,2),(0,-2)$
    \\
    $f(1,0)=-1$
    \\
    $f(2,0)=0$
    \\
    $f(0,2)=4$
    \\
    $f(0,-2)=4$
    \\
    There is an absolute min at $(1,0)$ and an absolute max at $(0,2)$ and $(0,-2)$






\item 14.8 \textbf{Lagrange Multipliers}
\\
\textit{Optimization:}
\\
To optimize $f(x,y,z)$ subject to $g(x,y,z)=k$ 
\\
1. Solve $\nabla f = \lambda \nabla g \; \; \; g=k$
\\
2. plug solutions into $f(x,y,z)$ and find the biggest/smallest

\\
\\
\textit{Example:}

Use Lagrange multipliers to find the extreme values of $f(x,y) = x^2 +2y^2$ on the circle $x^2+y^2 = 1$.  
    \\
    \textit{Answer:}
    \\
    $f(x,y)=x^2+2y^2 \; \; \; g(x,y)=x^2+y^2=1$
    \\
    $\nabla f= \langle 2x, 4y \rangle $
    \\
    $\nabla g= \langle 2x, 2y \rangle$
    \\
    $\langle 2x, 4y \rangle= \lambda \langle 2x, 2y \rangle$
    \\
    $2x= \lambda 2x \rightarrow \lambda=x$
    \\
    $4y= \lambda 2y \rightarrow y= lambda/2$
    \\
    $x^2+y^2 = 1 \rightarrow \lambda^2+\frac{\lambda}{2}^2=1 \rightarrow \lamda + \frac{\lambda}{2}=1 \rightarrow \frac{3 \lambda}{2}=1 \rightarrow \lambda= \frac{2}{3}$
    \\
    $x=\frac{2}{3} \; \; \; y= \frac{4}{3}$ or $(\frac{2}{3}, \frac{4}{3})$


\end{enumerate}


\end{document}

