\documentclass[11pt]{article}

\usepackage{epsfig}
\usepackage{amsfonts}
\usepackage{amssymb}
\usepackage{amstext}
\usepackage{amscd}
\usepackage{amsmath}
\usepackage{xspace}
\usepackage{theorem}
\usepackage{float}
\usepackage[table]{xcolor}
\usepackage{color}
\usepackage{pgfplots}

\definecolor{stainlessSteel}{cmyk}{0,0,0.02,0.12}
%\usepackage{layout}% if you want to see the layout parameters
                     % and now use \layout command in the body

% This is the stuff for normal spacing
\makeatletter
 \setlength{\textwidth}{6.5in}
 \setlength{\oddsidemargin}{0in}
 \setlength{\evensidemargin}{0in}
 \setlength{\topmargin}{0.25in}
 \setlength{\textheight}{8.25in}
 \setlength{\headheight}{0pt}
 \setlength{\headsep}{0pt}
 \setlength{\marginparwidth}{59pt}

 \setlength{\parindent}{0pt}
 \setlength{\parskip}{5pt plus 1pt}
 \setlength{\theorempreskipamount}{5pt plus 1pt}
 \setlength{\theorempostskipamount}{0pt}
 \setlength{\abovedisplayskip}{8pt plus 3pt minus 6pt}
 \setlength{\intextsep}{15pt plus 3pt minus 6pt}

 \renewcommand{\section}{\@startsection{section}{1}{0mm}%
                                   {2ex plus -1ex minus -.2ex}%
                                   {1.3ex plus .2ex}%
                                   {\normalfont\Large\bfseries}}%
 \renewcommand{\subsection}{\@startsection{subsection}{2}{0mm}%
                                     {1ex plus -1ex minus -.2ex}%
                                     {1ex plus .2ex}%
                                     {\normalfont\large\bfseries}}%
 \renewcommand{\subsubsection}{\@startsection{subsubsection}{3}{0mm}%
                                     {1ex plus -1ex minus -.2ex}%
                                     {1ex plus .2ex}%
                                     {\normalfont\normalsize\bfseries}}
 \renewcommand\paragraph{\@startsection{paragraph}{4}{0mm}%
                                    {1ex \@plus1ex \@minus.2ex}%
                                    {-1em}%
                                    {\normalfont\normalsize\bfseries}}
 \renewcommand\subparagraph{\@startsection{subparagraph}{5}{\parindent}%
                                       {2.0ex \@plus1ex \@minus .2ex}%
                                       {-1em}%
                                      {\normalfont\normalsize\bfseries}}
\makeatother

\newcounter{thelecture}

\newenvironment{proof}{{\bf Proof:  }}{\hfill\rule{2mm}{2mm}}
\newenvironment{proofof}[1]{{\bf Proof of #1:  }}{\hfill\rule{2mm}{2mm}}
\newenvironment{proofofnobox}[1]{{\bf#1:  }}{}
\newenvironment{example}{{\bf Example:  }}{\hfill\rule{0mm}{0mm}} % change 2mm 2mm for square

%\renewcommand{\theequation}{\thesection.\arabic{equation}}
%\renewcommand{\thefigure}{\thesection.\arabic{figure}}

\newtheorem{fact}{Fact}
\newtheorem{lemma}[fact]{Lemma}
\newtheorem{theorem}[fact]{Theorem}
\newtheorem{definition}[fact]{Definition}
\newtheorem{corollary}[fact]{Corollary}
\newtheorem{proposition}[fact]{Proposition}
\newtheorem{claim}[fact]{Claim}
\newtheorem{exercise}[fact]{Exercise}

% math notation
\newcommand{\R}{\ensuremath{\mathbb R}}
\newcommand{\Z}{\ensuremath{\mathbb Z}}
\newcommand{\N}{\ensuremath{\mathbb N}}
\newcommand{\B}{\ensuremath{\mathbb B}}
\newcommand{\F}{\ensuremath{\mathcal F}}
\newcommand{\SymGrp}{\ensuremath{\mathfrak S}}
\newcommand{\prob}[1]{\ensuremath{\text{{\bf Pr}$\left[#1\right]$}}}
\newcommand{\expct}[1]{\ensuremath{\text{{\bf E}$\left[#1\right]$}}}
\newcommand{\size}[1]{\ensuremath{\left|#1\right|}}
\newcommand{\ceil}[1]{\ensuremath{\left\lceil#1\right\rceil}}
\newcommand{\floor}[1]{\ensuremath{\left\lfloor#1\right\rfloor}}
\newcommand{\ang}[1]{\ensuremath{\langle{#1}\rangle}}
\newcommand{\poly}{\operatorname{poly}}
\newcommand{\polylog}{\operatorname{polylog}}

% anupam's abbreviations
\newcommand{\e}{\epsilon}
\newcommand{\half}{\ensuremath{\frac{1}{2}}}
\newcommand{\junk}[1]{}
\newcommand{\sse}{\subseteq}
\newcommand{\union}{\cup}
\newcommand{\meet}{\wedge}
\newcommand{\dist}[1]{\|{#1}\|_{\text{dist}}}
\newcommand{\hooklongrightarrow}{\lhook\joinrel\longrightarrow}
\newcommand{\embeds}[1]{\;\lhook\joinrel\xrightarrow{#1}\;}
\newcommand{\mnote}[1]{\normalmarginpar \marginpar{\tiny #1}}

%%%%%%%%%%%%%%%%%%%%%%%%%%%%%%%%%%%%%%%%%%%%%%%%%%%%%%%%%%%%%%%%%%%%%%%%%%%
% Document begins here %%%%%%%%%%%%%%%%%%%%%%%%%%%%%%%%%%%%%%%%%%%%%%%%%%%%
%%%%%%%%%%%%%%%%%%%%%%%%%%%%%%%%%%%%%%%%%%%%%%%%%%%%%%%%%%%%%%%%%%%%%%%%%%%

\newcommand{\hwheadings}[3]{
{\bf Calculus 200 -  Fall 2015} \hfill {{\bf Problem Set #1}}\\
{{\bf } #2} \hfill {{\bf Due:} #3} \\
\rule[0.1in]{\textwidth}{0.025in}
%\thispagestyle{empty}
}

\begin{document}

\hwheadings{6}{NAME}{Oct 28, 2015}


\begin{enumerate}


    
    
    
    \item Let $f(x,y) = \frac{x^3+y^3}{xy^2}$.  Set $y=mx$ and show that the resulting limit depends on $m$, and therefore the limit $\lim \limits_{(x,y) \to (0,0)} f(x,y)$ does not exist.
    \\
    \textbf{Answer:}
    Let $x=0$ and $y \rightarrow 0$ $ \frac{o^3+y^3}{0y^2}=\frac{y^2}{0}=$ Does not exist.
    \\
    Now let $x \rightarrow 0$ and $y=0$  $\frac{x^3+0^3}{x0^2}=\frac{x^3}{0}=$ Does not exist.
    \\
    Lets say $y=mx$ we substiute it in and get $ \frac{x^3+(mx)^3}{x(mx)^2}= \frac{x^3+m^3x^3}{xm^2x^2}=\frac{x^3(1+m^3)}{x^3m^2}=\frac{1+m^3}{m^2}$ We then see that $m$ can not be reduced from the equation, therefore the fraction is always dependent on $m$. Because $m=\frac{y}{x}$ we can see that no matter if $x$ or $y$ is equal to zero, the limit will not exist.
    
    \item Find the limit by using polar coordinates:
    
    \[ \lim_{(x,y) \to (0,0)}  \frac{\sqrt{x^2+y^2}}{\sin \sqrt{x^2+y^2}} \]
    \\
    \textbf{Answer:}
    From the defintion of polar equations, we see that $ \frac{\sqrt{x^2+y^2}}{\sin \sqrt{x^2+y^2}} = \frac{r}{\sin r}$ so we want to take the limit:
    \[
    \lim_{(r,\theta) \to (0,0)} \frac{r}{\sin r}
    \]
   Because there is no $\theta$ in the equation we can just say
   \[
    \lim_{r \to 0} \frac{r}{\sin r}
    \]
    as $r \rightarrow 0$ we see that the limit approuches $1$
    
    
    
    
   % \item Prove  or disprove the existence of the limit:
    %\[ \lim_{(x,y) \to ()} \]
    
    
    
    \item Let $f(x,y) = \frac{y^2}{(1+x^2)^3}$.  Calculate $f_x(1,3)$ and $f_y(1,3)$.
    \textbf{Answer:}
    We can say that $\frac{y^2}{(1+x^2)^3}= y^2(1+x^2)^{-3}$
    \\
    $f_x=-3y^2 2x (1+x^2)^{-2}=\frac{-3y^2 2x}{(1+x^2)^{2}}$
    \\
    $f_x(1,3)=\frac{-3(3)^2 2(1)}{(1+(1)^2)^{2}}=-13 \frac{1}{2}$
    \\
    $f_y=2y(1+x^2)^{-3}=\frac{2y}{(1+x^2)^3}$
    \\
    $f_y(1,3)=\frac{2(3)}{(1+(1)^2)^3}=\frac{3}{4}$
    
    
    \item Calculate $f_z(0,0,1,1)$ where $f(x,y,z,w) = \frac{e^{xz+y}}{z^2+w}$.
    \\
    $f_z=-e^{xyz+y}(z^2+w)^{-2}2z+xye^{xyz+y}(z^2+w)^{-1}= \frac{-e^{xyz+y}2z}{(z^2+w)^{2}}+\frac{xye^{xyz+y}}{z^2+w}$
    \\
    $f_z(0,0,1,1) \frac{-e^{xyz+y}2z}{(z^2+w)^{2}}+\frac{xye^{xyz+y}}{z^2+w}= \frac{-e^0 2}{2^2} + 0 = \frac{1}{2}$
    
    
    \item A process called \textbf{tag-and-recapture} is used to estimate populations of animals in the wild.  First, some number $T$ of the animals are captured, tagged and released into back into the wild.  Later, a number $N$ animals are captured, of which $t$ are observed to be tagged.  The estimate of the total population is then $P(T, N, t) = \frac{TN}{t}$.  Compute $P(100, 60, 15)$.  The proportion of the tagged animals in the recapture is $\frac{15}{60} = \frac{1}{4}$.  Based on this estimate of the total population, what proportion of the total population has been tagged?  Now computer $\frac{\partial P}{\partial t}(100, 60, 15)$ and use it to estimate how much your population estimate would change if one more recatured animal were tagged.
    \textbf{Answer:}
    $P(T,N,t)=\frac{TN}{t}$
    \\
    $P(100,60,15)= \frac{(100)(60)}{15}=400$
    \\
    $\frac{100}{400}=\frac{1}{4}$
    \\
    $\frac{\delta P}{\delta t}(100,60,15)=\frac{-TN}{t^2}=-\frac{(100)(60)}{15^2}=\frac{80}{3}$ Therefor it would go down by $\frac{80}{3}$
    \\
    This would mean that the total population would go down by about 26.
    \\
    If we plug in $P(100,60,16)$ we get $375$, which is down 25 from our original population, therefore our model is a sufficent estimate.
    
    
    
    
    
    \item Show the function $f_n(x,t) = \sin n\pi x \cos n \pi ct$ satisfy the \textbf{wave equation} 
    \[ c^2 \frac{\partial^2 f}{\partial x^2} = \frac{\partial^2 f}{\partial t^2},  \]
   for any positive integer $n$ and any constant $c$.
   \\
   
   \textbf{Answer:}
   \\
   $f_n(x,t) = \sin n\pi x \cos n \pi ct$
   \\
   $f_n= \sin (n \pi x)(- \pi c t \sin (n \pi ct))+ \cos(n \pi ct)(\pi x \cos (n \pi x))$
   \\
   $f_{nx}=-n\pi \cos (n \pi x)(- \pi c t \sin (n \pi c t))+ \cos (n \pi c t)(\pi \cos (n \pi x)- \pi^2 xn \sin (n \pi x))$
   
   \\
   $f_{nt}= \sin (n \pi x)(- \pi c \sin (n \pi ct)-n \pi^2 c^2 t \cos (n \pi ct))- n \pi c \sin (n \pi c t)(\pi x \cos (n \pi x)) $
   \\
   We want to show that $c^2f_{nx}=f_{nt}$
   \\
   $c^2(-n\pi \cos (n \pi x)(- \pi c t \sin (n \pi c t))+ \cos (n \pi c t)(\pi \cos (n \pi x)- \pi^2 xn \sin (n \pi x)))=(\sin (n \pi x)(- \pi c \sin (n \pi ct)-n \pi^2 c^2 t \cos (n \pi ct))- n \pi c \sin (n \pi c t)(\pi x \cos (n \pi x)))$
   \\
   \\
   $-c^2n\pi \cos (n \pi x)(- \pi c t \sin (n \pi c t))+ c^2 \cos (n \pi c t)(\pi \cos (n \pi x)- c^2 \pi^2 xn \sin (n \pi x))=\sin (n \pi x)(- \pi c \sin (n \pi ct)-n \pi^2 c^2 t \cos (n \pi ct))- n \pi c \sin (n \pi c t)(\pi x \cos (n \pi x))$
   \\
   This can be reduced to equal each other.
   
    
    \item Show that if $f(x)$ is a function with a continuous second derivative, then $f(x-ct)$ and $f(x+ct)$ are both solutions of the wave equation.  If $x$ represents position and $t$ represents time, explain why $c$ can be interpreted as teh velocity of the wave.  
    \\
    \textbf{Answer:}
    \\
    Prove that $f(x+ct)$ satisfies 
     \[ c^2 \frac{\partial^2 f}{\partial x^2} = \frac{\partial^2 f}{\partial t^2},  \]
     $f(x+ct) \; \; \; \;$ $f_x=f'(x+ct)$
     \\
     $f_{xx}=f''(x+ct)$
     \\
     $f_t=f'(x+ct)c$
     \\
     $f_{tt}=f''(x+ct)c^2$
     \\
     $c^2f''(xx+ct)=f''(x+ct)c^2 \checkmark$
    
    
    \item In a chemical reaction, the temperature $T$, entropy $S$, Gibbs free enger $G$, and enthalpy $H$ are related by $G = H - TS$.  Show that $\frac{\partial(G/T)}{\partial T} =  - \frac{H}{T^2}$.
    \\
    \textbf{Answer:}
    \\
    $G=H-TS$
    \\
    $(G/T)=\frac{H}{T}-S$
    \\
    $\frac{\partial(G/T)}{\partial T}=-\frac{H}{T^2}$
    
    
    \item Suppose that the concentration of some pollutant in a river as a function of position $x$ and time $t$ is given by $p(x,t) = p_0 (x-ct) e^{-\mu t}$ for constants $p_0, \mu,$ and  $c$.  Show that $\frac{\partial p}{\partial t} = -c \frac{\partial p}{\partial x} - \mu p$.  Interpret both $\frac{\partial p_0}{\partial x}$ and  $\frac{\partial p}{\partial t}$, and explain how this equation relates to the change in pollution at a specific location to the current of the river and the rate at which the pollutant decays.
    \\
    \textbf{Answer:}
    \\
     $p(x,t) = p_0 (x-ct) e^{-\mu t}$
     \\
     $\frac{\partial p}{\partial t}=p_0e^{-\mu t}(-c)+  p_0 -\mu (x-ct) e^{-\mu t}$
     \\
    $\frac{\partial p}{\partial x}=P_0(1)e^{-\mu t}   $
    \\
    We want to show that  $\frac{\partial p}{\partial t} = -c \frac{\partial p}{\partial x} - \mu p$ or $p_0e^{-\mu t}(-c)+  p_0 -\mu (x-ct) e^{-\mu t}=-c(P_0(1)e^{-\mu t})- \mu p_0$
    \\
    $p_0e^{-\mu t}(-c)+  p_0 -\mu (x-ct) e^{-\mu t}=p_0e^{-\mu t}(-c)+  p_0 -\mu (x-ct) e^{-\mu t} \checkmark$
    
    \item Suppose $f(x,y) = x^2 e^y$, $x(t) = t^2 - 1$, and $y(t) = \sin t$, find the derivative of $g(t) = f(x(t), y(t))$.
    \\
    \textbf{Answer:}
\\
$g(t)'=\frac{\partial g(t)}{\partial x} \frac{dx}{dt} + \frac{\partial g(t)}{\partial y} \frac{dy}{dt}$
\\
$\frac{\partial g(t)}{\partial x}= 2xe^y$
\\
$\frac{\partial g(t)}{\partial y}= x^2e^y$
\\
$\frac{dx}{dt}=2t$
\\
$\frac{dy}{dt}= \cos (t)$
\\
Therefore
\\
$g(t)'=(2xe^y)(2t)+(x^2e^y)(\cos (t)= 4txe^y+x^2e^y \cos (t)$




    \item The Environmental Protection Agency uses the 55/45 rule for combining a car's highway gas mileage rating $h$ and its city gas mileage rating $c$ into a single rating $R$ for fuel efficiency using the formula
    \[ 
        R = \frac{1}{0.55/c + 0.45/h}
    \]
    Find the first-order Taylor series for $R(c,h)$ about $(1,1)$.  Hint: Let $a$ and $b$ be constants; find Taylor series for $f(u) = R(c+au, h+au)$ about $u=0$. 
    \\
    \textbf{Answer:}
    \\
    $T=R(1,1)+R_c(1,1)(c-1)+R_h(1,1)(h-1)$
    \\
    $R(1,1)=\frac{1}{0.55/1 + 0.45/1}=\frac{1}{1}=1$
    \\
    $R_c=\frac{-1}{-.55/c^2 (.55/c +.45/h)^2}$
    \\
    $R_c(1,1)=\frac{-1}{-.55/1^2 - (.55/1 +.45/1)^2}=\frac{-1}{-.55  (.55 +.45)^2}= \frac{-1}{-.55(1)}= \frac{1}{.55} $
    \\
    $R_h= \frac{-1}{-.45/h^2 (.55/c +.45/h)^2} $
    \\
    $R_h(1,1)=\frac{-1}{-.45/1^2 (.55/1+.45/1)^2}= \frac{-1}{-.45 (.55 +.45)^2}=\frac{1}{.45}$
    \\
    $T= \frac{1}{.55}(c-1)+\frac{1}{.45}(h-1)$
    \\
    $T=\frac{1}{.55}c - \frac{1}{.55} + \frac{1}{.45}h -\frac{1}{.45}$
   
    
    
    
    
    
    
    \item Suppose $f(x,y) = e^{xy}$, $x(u,v) = 3u \sin v$ and $y(u,v) = 4v^2u$.  For $g(u,v) = f(x(u,v), y(u,v))$, find the partial derivatives $\frac{\partial g}{\partial u}$ and $\frac{\partial g}{\partial v}$.
    \\
    \textbf{Answer:}
    \\
    $\frac{\partial g}{\partial u}=\frac{\partial g}{\partial x} \frac{\partial x}{\partial u}+ \frac{\partial g}{\partial y} \frac{\partial y}{\partial u}$
    \\
    $\frac{\partial g}{\partial x}=ye^{xy}$
    \\
    $\frac{\partial x}{\partial u}=3 \sin (v)$
    \\
    $\frac{\partial g}{\partial y}=xe^{xy}$
    \\
    $\frac{\partial y}{\partial u}=4v^2$
    \\
    $\frac{\partial g}{\partial u}=(ye^{xy})(3 \sin (v))+(=xe^{xy})(4v^2)$
    
    \\
    \\
    $\frac{\partial g}{\partial v}=\frac{\partial g}{\partial x}\frac{\partial x}{\partial v}+\frac{\partial g}{\partial y}\frac{\partial y}{\partial v}$
    \\
    $\frac{\partial g}{\partial x}=ye^{xy}$
    \\
    $\frac{\partial x}{\partial v}=3u \cos(v)$
    \\
    $\frac{\partial g}{\partial y}=xe^{xy}$
    \\
    $\frac{\partial y}{\partial v}=8vu$
    \\
    $\frac{\partial g}{\partial v}=(ye^{xy})(3u \cos(v))+ (xe^{xy}) (8vu) $
    
    
    
    
    \item Find the directional derivative $D_{{\bf u}}f(1, -1)$ when $f(x,y) = x^2 + y^2$ for ${\bf u}$ in the direction $\langle -3, 4 \rangle$. 
    \\
    \textbf{Answer:}
    \\
    Make the direction vector into a directional unit vector by setting ${\bf u}=\frac{\bf u}{||u||}$
    \\
    $\langle \frac{-3}{5}, \frac{4}{5} \rangle$
    \\
    
    $D_{{\bf u}}f(a, b)=f_x u_1+f_y u_2$ where $u_1,u_2$ at the components of the direction unit vector. Therefore:
    \\
    $D_{{\bf u}}f(1, -1)=f_x(\frac{-3}{5})+f_y(\frac{4}{5})$
    \\
    $f_x=2x$
    \\
    $f_y=2y$
    \\
    $D_{{\bf u}}f(1, -1)=2x(\frac{-3}{5})+2y(\frac{4}{5})=\frac{-6}{5}x+\frac{8}{5}y$
    
\end{enumerate}
\end{document}


