\documentclass[11pt]{article}

\usepackage{epsfig}
\usepackage{amsfonts}
\usepackage{amssymb}
\usepackage{amstext}
\usepackage{amscd}
\usepackage{amsmath}
\usepackage{xspace}
\usepackage{theorem}
\usepackage{float}
\usepackage[table]{xcolor}
\usepackage{color}
\usepackage{pgfplots}

\definecolor{stainlessSteel}{cmyk}{0,0,0.02,0.12}
%\usepackage{layout}% if you want to see the layout parameters
                     % and now use \layout command in the body

% This is the stuff for normal spacing
\makeatletter
 \setlength{\textwidth}{6.5in}
 \setlength{\oddsidemargin}{0in}
 \setlength{\evensidemargin}{0in}
 \setlength{\topmargin}{0.25in}
 \setlength{\textheight}{8.25in}
 \setlength{\headheight}{0pt}
 \setlength{\headsep}{0pt}
 \setlength{\marginparwidth}{59pt}

 \setlength{\parindent}{0pt}
 \setlength{\parskip}{5pt plus 1pt}
 \setlength{\theorempreskipamount}{5pt plus 1pt}
 \setlength{\theorempostskipamount}{0pt}
 \setlength{\abovedisplayskip}{8pt plus 3pt minus 6pt}
 \setlength{\intextsep}{15pt plus 3pt minus 6pt}

 \renewcommand{\section}{\@startsection{section}{1}{0mm}%
                                   {2ex plus -1ex minus -.2ex}%
                                   {1.3ex plus .2ex}%
                                   {\normalfont\Large\bfseries}}%
 \renewcommand{\subsection}{\@startsection{subsection}{2}{0mm}%
                                     {1ex plus -1ex minus -.2ex}%
                                     {1ex plus .2ex}%
                                     {\normalfont\large\bfseries}}%
 \renewcommand{\subsubsection}{\@startsection{subsubsection}{3}{0mm}%
                                     {1ex plus -1ex minus -.2ex}%
                                     {1ex plus .2ex}%
                                     {\normalfont\normalsize\bfseries}}
 \renewcommand\paragraph{\@startsection{paragraph}{4}{0mm}%
                                    {1ex \@plus1ex \@minus.2ex}%
                                    {-1em}%
                                    {\normalfont\normalsize\bfseries}}
 \renewcommand\subparagraph{\@startsection{subparagraph}{5}{\parindent}%
                                       {2.0ex \@plus1ex \@minus .2ex}%
                                       {-1em}%
                                      {\normalfont\normalsize\bfseries}}
\makeatother

\newcounter{thelecture}

\newenvironment{proof}{{\bf Proof:  }}{\hfill\rule{2mm}{2mm}}
\newenvironment{proofof}[1]{{\bf Proof of #1:  }}{\hfill\rule{2mm}{2mm}}
\newenvironment{proofofnobox}[1]{{\bf#1:  }}{}
\newenvironment{example}{{\bf Example:  }}{\hfill\rule{0mm}{0mm}} % change 2mm 2mm for square

%\renewcommand{\theequation}{\thesection.\arabic{equation}}
%\renewcommand{\thefigure}{\thesection.\arabic{figure}}

\newtheorem{fact}{Fact}
\newtheorem{lemma}[fact]{Lemma}
\newtheorem{theorem}[fact]{Theorem}
\newtheorem{definition}[fact]{Definition}
\newtheorem{corollary}[fact]{Corollary}
\newtheorem{proposition}[fact]{Proposition}
\newtheorem{claim}[fact]{Claim}
\newtheorem{exercise}[fact]{Exercise}

% math notation
\newcommand{\R}{\ensuremath{\mathbb R}}
\newcommand{\Z}{\ensuremath{\mathbb Z}}
\newcommand{\N}{\ensuremath{\mathbb N}}
\newcommand{\B}{\ensuremath{\mathbb B}}
\newcommand{\F}{\ensuremath{\mathcal F}}
\newcommand{\SymGrp}{\ensuremath{\mathfrak S}}
\newcommand{\prob}[1]{\ensuremath{\text{{\bf Pr}$\left[#1\right]$}}}
\newcommand{\expct}[1]{\ensuremath{\text{{\bf E}$\left[#1\right]$}}}
\newcommand{\size}[1]{\ensuremath{\left|#1\right|}}
\newcommand{\ceil}[1]{\ensuremath{\left\lceil#1\right\rceil}}
\newcommand{\floor}[1]{\ensuremath{\left\lfloor#1\right\rfloor}}
\newcommand{\ang}[1]{\ensuremath{\langle{#1}\rangle}}
\newcommand{\poly}{\operatorname{poly}}
\newcommand{\polylog}{\operatorname{polylog}}

% anupam's abbreviations
\newcommand{\e}{\epsilon}
\newcommand{\half}{\ensuremath{\frac{1}{2}}}
\newcommand{\junk}[1]{}
\newcommand{\sse}{\subseteq}
\newcommand{\union}{\cup}
\newcommand{\meet}{\wedge}
\newcommand{\dist}[1]{\|{#1}\|_{\text{dist}}}
\newcommand{\hooklongrightarrow}{\lhook\joinrel\longrightarrow}
\newcommand{\embeds}[1]{\;\lhook\joinrel\xrightarrow{#1}\;}
\newcommand{\mnote}[1]{\normalmarginpar \marginpar{\tiny #1}}

%%%%%%%%%%%%%%%%%%%%%%%%%%%%%%%%%%%%%%%%%%%%%%%%%%%%%%%%%%%%%%%%%%%%%%%%%%%
% Document begins here %%%%%%%%%%%%%%%%%%%%%%%%%%%%%%%%%%%%%%%%%%%%%%%%%%%%
%%%%%%%%%%%%%%%%%%%%%%%%%%%%%%%%%%%%%%%%%%%%%%%%%%%%%%%%%%%%%%%%%%%%%%%%%%%

\newcommand{\hwheadings}[3]{
{\bf Calculus 200 -  Fall 2015} \hfill {{\bf Problem Set #1}}\\
{{\bf } #2} \hfill {{\bf Due:} #3} \\
\rule[0.1in]{\textwidth}{0.025in}
%\thispagestyle{empty}
}

\begin{document}

\hwheadings{7}{Chelsea Blake}{Oct 10, 2015}


\begin{enumerate}


    
    
    
    \item Find the directional derivative of $f(x,y) = x^3y^4+x^4y^3$ at the point $(1,1)$ in the direction $\theta =\pi/6$.
    \\
    \textbf{Answer:}
    \\
    $D_uf(a,b)= \langle f_x, f_y \rangle \cdot \langle u_1,u_2 \rangle$
    \\
    $\nabla f \cdot u= ||f|| \times ||u|| \times \cos \theta$
    \\
    $f_x=3x^2y^4+4x^3y^3$
    \\
    $f_y=4y^3x^3+3y^2x^4$
    \\
    $\nabla f= \langle 3x^2y^4+4x^3y^3, 4y^3x^3+3y^2x^4 \rangle$
    \\
    $\nabla f(1,1)= \langle 3(1)^2(1)^4+4(1)^3(1)^3, 4(1)^3(1)^3+3(1)^2(1)^4 \rangle$
    \\
     $\nabla f(1,1)= \langle 7,7 \rangle$
     \\
     $||\nabla f(1,1)||=\sqrt{98}$
     \\
     $||u||=1$ because $||u||$ is a unit vector.
     \\
     $D_uf(1,1)= \sqrt{98} \times 1 \times \cos{\pi/6}= \sqrt{98} \times \frac{\sqrt{3}}{2}= \frac{\sqrt{294}}{2} $
    
   
   \item  Find the shortest distance from point $(1,0,-2)$ to the plane $x+2y+z=4$.
   \\
   \textbf{Answer:}
   \\
   Distance function: $d(1,0,-2),(x,y,z)=\sqrt{(x-1)^2+(y)^2+(z+2)^2}$
   \\
   $z=-x-2y+4$
   \\
   $d(1,0,-2),(x,y,z)=\sqrt{(x-1)^2+(y)^2+(-x-2y+6)^2}$
   \\
   $d(1,0,-2),(x,y,z)=\sqrt{x^2-2x+1+y^2+x^2+4y^2+4xy-12x-24y+36}$
   \\
   $d(1,0,-2),(x,y,z)=\sqrt{2x^2+5y^2+4xy-14x-24y+37}$
   \\
   $d^2=2x^2+5y^2+4xy-14x-24y+37$
   \\
   $f_x=4x+4y-14=0 \rightarrow x=\frac{7}{2}-y$
   \\
   $f_y=10y+4x-24=0 \rightarrow 10y-4y-10=0$
   \\
  $6y=10 \rightarrow y=\frac{10}{6}$
  \\
  $x=\frac{7}{2}-\frac{10}{6}= \frac{11}{6}$
  \\
  Check to see if min:
  \\
  $f_{xx}= 4$
  \\
  $f_{yy}= 10$
  \\
  $f_{xy}= 4$
  \\
  $D=(4)(10)-(16)=24>0 \; \; \; f_{xx}>0$ Therefore $(\frac{11}{6}, \frac{10}{6})$
   
   
   \item Find and classify the critical points of $f(x,y) = 10x^2y - 5x^2 - 4y^2-x^4 - 2y^4$.  
   \\
   \textbf{Answer:}
   \\
   $f_x=20xy-10x-4x^3=0$
   \\
   $f_y=10x^2-8y-8y^3=0 \rightarrow x=\sqrt{\frac{-8y^3-8y}{10}}$
   \\
   $20(\sqrt{\frac{-8y^3-8y}{10}})y-10(\sqrt{\frac{-8y^3-8y}{10}})-4(\sqrt{\frac{-8y^3-8y}{10}})^3=0$
   \\
   $y=0$
   \\
   $x=0$
   \\
   $f_{x,x}=20y-10-12x^2$
   \\
   $f_{y,y}=-8-24y^2$
   \\
   $f_{x,y}=20x$
   \\
   $f_{xx}(0,0)f_{yy}(0,0)-f_{xy}(0,0)^2=-10+8=18 >0 $
   \\$f_{xx}(0,0)=-10 < 0$
   \\
   $(0,0)$ is a max.
   
   
   \item Prove that $f$ is a differentiable function of two or three variables then the maximum value of the directional derivative $D_{\bf u}f({\bf x})$ is $| \nabla f({\bf x})|$ and occurs when ${\bf u}$ is in the direction of the gradient vector.
   \\
   \textbf{Answer:}
   \\
   We know that $D_uf(a,b)= \langle f_x, f_y \rangle \cdot \langle u_1,u_2 \rangle$ or $D_uf(a,b)= ||\langle f_x, f_y \rangle|| \times ||u|| \times \cos \theta$ We also know that $||u||=1$ because $u$ is the unit vector. Finally, we know that $-1 \ge \cos \theta \le 1$.
   \\
   We can simplify $D_uf(a,b)$ to $D_uf(a,b)= ||\nabla f(x)|| \times \cos \theta$  
   \\
   If we take the max of $D_uf(a,b)$ we will see that the max of $\cos \theta =1$ and therefore the max of $D_uf(a,b)= ||\nabla f(x)||$. 
   \\
   This max occurs when the $\cos \theta =1$ and we know this happens when $\theta = 0$. $\theta$ is the angle between $u$ and the gradient vector, therefore if $u$ is in the direction of the gradient vector $\theta$ will be $0$, the $\cos \theta =1$ and $D_uf(a,b)= ||\nabla f(x)||$. 
   
   \item If $f(x,y) = xe^y$, find the rate of change of $f$ at the point $P(2,0)$ in the direction from $P$ to $Q(\frac{1}{2}, 2)$.  In what direction does $f$ have the maximum rate of change?  What is the maximum rate of change?
   \\
   \textbf{Answer:}
   $D_uf(a,b)= \langle f_x,f_y \rangle \cdot \langle u_1, u_2 \rangle$
   \\
   $f_x(2,0)=e^y=e^0=1$
   \\
   $f_y(2,0)=xe^y=(2)e^0=2$
   \\
   $D_uf(0,0)=\langle 1,2 \rangle \cdot \langle \frac{1}{2},2 \rangle$
   \\
   $D_uf(0,0)=1 \times \frac{1}{2}+ 2 \times 2= \frac{5}{2} $
   \\
   When $u$ is in the same direction as the vector (i.e. when $u$ is in the direction of $p$) then the angle between them is zero, and by the equation seen on question \textbf{4} $D_uf(a,b)= ||\nabla f(x)|| \times \cos \theta$ we know that if the angle is zero, $f$ will be at it's max. This max will be $D_uf(a,b)= ||\nabla f(x)||=|| 1,2||=\sqrt{5}$.
 
   
   
   
   
   
   
   
   \item Given that the directional derivative of $f$ at $(x, y)$ in the direction of a unit vector ${\bf u} = \langle a, b \rangle$ is 
   \[D_{\bf u} f(x,y) = \lim_{h \to 0} \frac{f(x+ha, y +hb) - f(x, y)}{h} \]
   if the limit exists, prove that
   \[D_{\bf u} f(x,y)  = f_x(x,y) a + f_y(x,y) b \]
   
   \textbf{Answer:}
   \\
   $D_{\bf u} f(x,y) = \lim_{h \to 0} \frac{f(x+ha, y +hb) - f(x, y)}{h}$
   
   
   
   
   
    
    
    
    \item Find the absolute maximum and minimum values of $f(x,y) = x^2 +y^2 - 2x$ on the closed triangle region with vertices $(2,0), (0,2)$, and $(0,-2)$.
    \\
    \textbf{Answer:}
    \\
    $f_x=2x-2=0 \; \; \; x=1$
    \\
    $f_y=2y \; \; \; y=0$
    \\
    $(1,0)$
    \\
    E.P.: $(1,0),(2,0),(0,2),(0,-2),(0,0)$
    \\
    $f(0,y)=y^2$
    \\
    $f(x,x+2)=x^2+(x=2)^2-2x \rightarrow 2x^2+2x+4 \rightarrow 4x+2=0 x=-\frac{2}{4} \; \; y= $
    \\
    $f(1,0)=-1$
    \\
    $f(2,0)=0$
    \\
    $f(0,2)=4$
    \\
    $f(0,-2)=4$
    \\
    $f(0,0)=0$
    There is an absolute min at $(1,0)$ and an absolute max at $(0,2)$ and $(0,-2)$
    
    \item A rectangular box without a lid is to be made from $12 \, m^2$ of cardboard.  Find the maximum volume of such a box.
    \\
    \textbf{Answer:}
    \\
    $v=l \times w \times h $ surface area must equal $12$ so $2lw+2lh+2wh=12$
    \\
    $f=l \times w \times h$ and $g=2lw+2lh+2wh=12$
    \\
    Find the gradien vectors of $f$ and $g$
    \\
    $\nabla f= \langle w \times h,l \times h, l \times w \rangle$
    \\
    $\nabla g= \langle 2w+2h, 2l+2h, 2l+2w \rangle $
    \\
    Plug into $\nabla f= \lambda \nabla g$
    \\
    $\langle w \times h,l \times h, l \times w \rangle= \lambda \langle 2w+2h, 2l+2h, 2l+2w \rangle$
    \\
    $w \times h= \lambda 2w+ \lambda 2h \rightarrow w(h- 2 \lambda)= \lambda 2h \rightarrow w= \frac{\lambda 2 h}{h-\lambda 2} \rightarrow w= \frac{\lambda 2 (\frac{\lambda 2 l}{1-2 \lambda})}{(\frac{\lambda 2 l}{1-2 \lambda})-\lambda 2} \rightarrow w= \frac{\lambda 2 (l- \lambda 2)^2}{(\lambda 2l)^2-\lambda 2} \rightarrow  w= \frac{l^2\lambda 2-8l\lambda^2-4 \lambda^3 }{4 \lambda^2l^2-\lambda^2}$
    \\
    $l \times h= \lambda 2l + \lambda 2h \rightarrow h(l-2 \lambda)=\lambda 2 l \rightarrow h= \frac{\lambda 2 l}{1-2 \lambda}$
    \\
    $l \times w= \lambda 2l + \lambda 2w \rightarrow l(h- \lambda 2)=\lambda 2w \rightarrow l= \frac{\lambda 2w}{w- \lambda 2}$
    \\
    $2lw+2lh+2wh=12 \rightarrow 2(\frac{\lambda 2w}{w- \lambda 2})(\frac{\lambda 2 h}{h-\lambda 2})+2(\frac{\lambda 2w}{w- \lambda 2})(\frac{\lambda 2 l}{1-2 \lambda})+2(\frac{\lambda 2 h}{h-\lambda 2})(\frac{\lambda 2 l}{1-2 \lambda})=12 \rightarrow \frac{8 \lmbda^2 wh}{wh-w\lambda 2-h \lambda 2 - 4 \lambda^2}+\frac{8 \lmbda^2 wl}{wl-w\lambda 2-l \lambda 2 - 4 \lambda^2}+ \frac{8 \lmbda^2 hl}{hl-h\lambda 2-l \lambda 2 - 4 \lambda^2}=12$
    \\
    Solve for length, width, and height then plug in to find volume.
    
    
    
    
    
    
    
    \item A scientist collects $n$ data points from two related quantities, i.e., $(x_i, y_i)$.  Suppose the scientist decides that the data are to be modeled using a linear equation $y=mx+b$.  Let $d_i = y_i - (mx_i +b)$ be the vertical deviation from the point $(x_i, y_i)$ from the linear model.  The method of least squares determines the model parameters $m$ and $b$ so as to minimize the sum of the square deviations, i.e.,  $\sum \limits_{i=1}^n d_i^2$.  Use calculus to derive the ``normal" equations and then solve for the parameters $m$ and $b$.
    \\
    \textbf{Answer:}
    \\
    
    
    
    \item Use Lagrange multipliers to find the extreme values of $f(x,y) = x^2 +2y^2$ on the circle $x^2+y^2 = 1$.  
    \\
    \textbf{Answer:}
    \\
    $f(x,y)=x^2+2y^2 \; \; \; g(x,y)=x^2+y^2=1$
    \\
    $\nabla f= \langle 2x, 4y \rangle $
    \\
    $\nabla g= \langle 2x, 2y \rangle$
    \\
    $\langle 2x, 4y \rangle= \lambda \langle 2x, 2y \rangle$
    \\
    $2x= \lambda 2x \rightarrow \lambda=x$
    \\
    $4y= \lambda 2y \rightarrow y= lambda/2$
    \\
    $x^2+y^2 = 1 \rightarrow \lambda^2+\frac{\lambda}{2}^2=1 \rightarrow \lamda + \frac{\lambda}{2}=1 \rightarrow \frac{3 \lambda}{2}=1 \rightarrow \lambda= \frac{2}{3}$
    \\
    $x=\frac{2}{3} \; \; \; y= \frac{4}{3}$ or $(\frac{2}{3}, \frac{4}{3})$
    
    
    
\end{enumerate}
\end{document}


