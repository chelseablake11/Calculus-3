\documentclass[11pt]{article}

\usepackage{epsfig}
\usepackage{amsfonts}
\usepackage{amssymb}
\usepackage{amstext}
\usepackage{amscd}
\usepackage{amsmath}
\usepackage{xspace}
\usepackage{theorem}
\usepackage{float}
\usepackage[table]{xcolor}
\usepackage{color}
\usepackage{pgfplots}

\definecolor{stainlessSteel}{cmyk}{0,0,0.02,0.12}
%\usepackage{layout}% if you want to see the layout parameters
                     % and now use \layout command in the body

% This is the stuff for normal spacing
\makeatletter
 \setlength{\textwidth}{6.5in}
 \setlength{\oddsidemargin}{0in}
 \setlength{\evensidemargin}{0in}
 \setlength{\topmargin}{0.25in}
 \setlength{\textheight}{8.25in}
 \setlength{\headheight}{0pt}
 \setlength{\headsep}{0pt}
 \setlength{\marginparwidth}{59pt}

 \setlength{\parindent}{0pt}
 \setlength{\parskip}{5pt plus 1pt}
 \setlength{\theorempreskipamount}{5pt plus 1pt}
 \setlength{\theorempostskipamount}{0pt}
 \setlength{\abovedisplayskip}{8pt plus 3pt minus 6pt}
 \setlength{\intextsep}{15pt plus 3pt minus 6pt}

 \renewcommand{\section}{\@startsection{section}{1}{0mm}%
                                   {2ex plus -1ex minus -.2ex}%
                                   {1.3ex plus .2ex}%
                                   {\normalfont\Large\bfseries}}%
 \renewcommand{\subsection}{\@startsection{subsection}{2}{0mm}%
                                     {1ex plus -1ex minus -.2ex}%
                                     {1ex plus .2ex}%
                                     {\normalfont\large\bfseries}}%
 \renewcommand{\subsubsection}{\@startsection{subsubsection}{3}{0mm}%
                                     {1ex plus -1ex minus -.2ex}%
                                     {1ex plus .2ex}%
                                     {\normalfont\normalsize\bfseries}}
 \renewcommand\paragraph{\@startsection{paragraph}{4}{0mm}%
                                    {1ex \@plus1ex \@minus.2ex}%
                                    {-1em}%
                                    {\normalfont\normalsize\bfseries}}
 \renewcommand\subparagraph{\@startsection{subparagraph}{5}{\parindent}%
                                       {2.0ex \@plus1ex \@minus .2ex}%
                                       {-1em}%
                                      {\normalfont\normalsize\bfseries}}
\makeatother

\newcounter{thelecture}

\newenvironment{proof}{{\bf Proof:  }}{\hfill\rule{2mm}{2mm}}
\newenvironment{proofof}[1]{{\bf Proof of #1:  }}{\hfill\rule{2mm}{2mm}}
\newenvironment{proofofnobox}[1]{{\bf#1:  }}{}
\newenvironment{example}{{\bf Example:  }}{\hfill\rule{0mm}{0mm}} % change 2mm 2mm for square

%\renewcommand{\theequation}{\thesection.\arabic{equation}}
%\renewcommand{\thefigure}{\thesection.\arabic{figure}}

\newtheorem{fact}{Fact}
\newtheorem{lemma}[fact]{Lemma}
\newtheorem{theorem}[fact]{Theorem}
\newtheorem{definition}[fact]{Definition}
\newtheorem{corollary}[fact]{Corollary}
\newtheorem{proposition}[fact]{Proposition}
\newtheorem{claim}[fact]{Claim}
\newtheorem{exercise}[fact]{Exercise}

% math notation
\newcommand{\R}{\ensuremath{\mathbb R}}
\newcommand{\Z}{\ensuremath{\mathbb Z}}
\newcommand{\N}{\ensuremath{\mathbb N}}
\newcommand{\B}{\ensuremath{\mathbb B}}
\newcommand{\F}{\ensuremath{\mathcal F}}
\newcommand{\SymGrp}{\ensuremath{\mathfrak S}}
\newcommand{\prob}[1]{\ensuremath{\text{{\bf Pr}$\left[#1\right]$}}}
\newcommand{\expct}[1]{\ensuremath{\text{{\bf E}$\left[#1\right]$}}}
\newcommand{\size}[1]{\ensuremath{\left|#1\right|}}
\newcommand{\ceil}[1]{\ensuremath{\left\lceil#1\right\rceil}}
\newcommand{\floor}[1]{\ensuremath{\left\lfloor#1\right\rfloor}}
\newcommand{\ang}[1]{\ensuremath{\langle{#1}\rangle}}
\newcommand{\poly}{\operatorname{poly}}
\newcommand{\polylog}{\operatorname{polylog}}

% anupam's abbreviations
\newcommand{\e}{\epsilon}
\newcommand{\half}{\ensuremath{\frac{1}{2}}}
\newcommand{\junk}[1]{}
\newcommand{\sse}{\subseteq}
\newcommand{\union}{\cup}
\newcommand{\meet}{\wedge}
\newcommand{\dist}[1]{\|{#1}\|_{\text{dist}}}
\newcommand{\hooklongrightarrow}{\lhook\joinrel\longrightarrow}
\newcommand{\embeds}[1]{\;\lhook\joinrel\xrightarrow{#1}\;}
\newcommand{\mnote}[1]{\normalmarginpar \marginpar{\tiny #1}}

%%%%%%%%%%%%%%%%%%%%%%%%%%%%%%%%%%%%%%%%%%%%%%%%%%%%%%%%%%%%%%%%%%%%%%%%%%%
% Document begins here %%%%%%%%%%%%%%%%%%%%%%%%%%%%%%%%%%%%%%%%%%%%%%%%%%%%
%%%%%%%%%%%%%%%%%%%%%%%%%%%%%%%%%%%%%%%%%%%%%%%%%%%%%%%%%%%%%%%%%%%%%%%%%%%

\newcommand{\hwheadings}[3]{
{\bf Calculus 200 -  Fall 2015} \hfill {{\bf Problem Set #1}}\\
{{\bf } #2} \hfill {{\bf Due:} #3} \\
\rule[0.1in]{\textwidth}{0.025in}
%\thispagestyle{empty}
}

\begin{document}

\hwheadings{4}{Chelsea Blake}{Oct 2, 2015}


\begin{enumerate}

 \item Find the limit: $$ \lim_{t \to 0} \left( e^{-3t} {\bf i} + \frac{t^2}{\sin^2(t)} {\bf j} + \cos (2t) {\bf k}  \right)$$
 \\
 Answer:
$\langle 1, \lim_{t \to 0}  \frac{t^2}{\sin^2(t)}, 1 \rangle  $
\\
$\langle 1, 1, 1 \rangle  $



\item Prove $$\frac{d}{dx}[f(t) {\bf r}(t)] = f'(t) {\bf r}(t)  + f(t) {\bf r}'(t) $$   

Answer: let ${\bf r}(t)= \langle a(t),b(t),c(t) \rangle$
\\
$[f(t){\bf r}(t)]'=[\langle f(t)a(t),f(t)b(t),f(t)c(t)]'$
\\
$\langle f'(t)a(t)+f(t)a'(t),f'(t)b(t)+f(t)b'(t),f'(t)c(t)+f(t)c'(t)$
\\
$\langle f'(t)a(t),f'(t)b(t),f'(t)c(t)\rangle + f(t)\lange a'(t),b'(t),c'(t)\rangle$
\\
$f'(t)\lange a(t),b(t),c(t)\rangle+ f(t)\lange a'(t),b'(t),c'(t)\rangle$
\\
$f'(t)  {\bf r}(t) + f(t) {\bf r}'(t)$



\item Prove $$\frac{d}{dx}[{\bf r}(t) \times {\bf s}(t)] = {\bf r}'(t) \times {\bf s}(t) + {\bf r}(t) \times {\bf s}'(t)$$    





\item Find the derivative of
\\the vector valued function ${\bf r}(t) = \langle t^4, \sqrt{t+1}, 3/t^2 \rangle$.
Answer:
\\
$ {\bf r}'(t)= \langle 4t^3,\frac{1}{2}(t+1)^{-\frac{1}{2}}, \frac{-6}{t^3} \rangle$




    \item Find the domain of the vector valued functions:
    \begin{enumerate}
        \item ${\bf r}(t) = \langle \sqrt{4-t^2}, e^{-3t}, \ln(t+1) \rangle$
        \\
        Answer: $\langle t<2, \infty, t \ge 0 \rangle$ 
        \\
        Domain: $[0,2)$
        \item ${\bf r}(t) = \langle t^2-t, \frac{\sin t}{\ln t}, 5 \rangle$
        \\
        Answer: $\langle \infty, t > 0, -- \rangle$
        \\
        Domain: $(0, \infty]$
        
       % \item ${\bf r}(t) = \langle \frac{1}{\cos \pi t}, \frac{\sin t}{t}, \frac{\ln(t)}{\sqrt{t} \rangle$
        \item ${\bf r}(t) = \langle t^4, \sqrt{t+1}, 3/t^2 \rangle$
        \\
        Answer: $\langle \infty, t \ge -1, \infty \rangle$
        \\
        Domain $[-1, \infty]$
    \end{enumerate}



\item Find the derivative of the vector valued functions:
    \begin{enumerate}
        \item ${\bf r}(t) = \langle \sin t, \sin t^2, \cos t \rangle$
        \\
        Answer: ${\bf r}'(t) = \langle \cos t, 2t \cos t^2, -\sin t \rangle$
        \item ${\bf r}(t) = \langle t^2-t, e^{t^2}, \sec 2t \rangle$
        \\
        Answer: ${\bf r}'(t) = \langle 2t-1, 2te^{t^2}, 2 \tan (2t) \sec (2t) \rangle $
        \\
        %\item ${\bf r}(t) = \langle \frac{1}{\cos \pi t}, \frac{\sin t}{t}, \frac{\ln(t)}{\sqrt{t} \rangle$
        \item ${\bf r}(t) = \langle t^4, \sqrt{t+1}, 3/t^2 \rangle$
        \\
        Answer: ${\bf r}'(t) = \langle 4t^3, \frac{1}{2}(t+1)^{-\frac{1}{2}}\rangle$
    \end{enumerate}
    
    
    
    \item Prove $\frac{d}{dt} [c {\bf r}(t) ] = c {\bf r}'(t)$ for  ${\bf r}(t) = \langle f(t), g(t), h(t) \rangle$.
    \\
    Answer: 
    \\
    $ [c {\bf r}(t) ]'= c' {\bf r}(t) +   c {\bf r}'(t) $ (see proof in problem 2)
    \\
    We know that $c$ is a constant and there derivative of a constant is $0$ so we get: 
    \\
    $0 {\bf r}(t) +   c {\bf r}'(t)=  c {\bf r}'(t)$
    
    %\item Prove $\frac{d}{dt} [f(t) {\bf r}(t) ] = f'(t) {\bf r}(t) + f(t) {\bf r}'(t)$ for  ${\bf r}(t) = \langle a(t), b(t), c(t) \rangle$.
    
    \item Given ${\bf r}_1(t) = \langle t^2, t^3, t \rangle$ and ${\bf r}_2(t) = \langle e^{3t}, e^{2t}, e^t \rangle$, find 
    \begin{enumerate}
        \item $\frac{d}{dt} [t^4 {\bf r}_1(t)]$
        \\
        Answer: $\frac{d}{dt} [t^4 \langle t^2, t^3, t \rangle]$
        \\
        $\frac{d}{dt} [\langle t^6, t^7, t^5 \rangle]$
        \\
        $\frac{d}{dt}= \langle 6t^5, 7t^6, 5t^4 \rangle]$
        \item $\frac{d}{dt}  [{\bf r}_1(t) \cdot {\bf r}_2(t)]$
        \\
        Answer: $\frac{d}{dt} [\langle t^2, t^3, t \rangle \cdot \langle e^{3t}, e^{2t}, e^t \rangle] $
\\
$\frac{d}{dt} [\langle t^2e^{3t}, t^3e^{2t}, te^t \rangle $
\\
$\frac{d}{dt}=\langle 3t^2e^{3t}+2te^{3t}, 2t^3e^{2t}+3t^2e^{2t}, te^t+e^t \rangle$
        
        \item $\frac{d}{dt}  [{\bf r}_1(t) \times {\bf r}_2(t)]  $
        \\
        Answer: $\frac{d}{dt} [\langle t^2, t^3, t \rangle \times \langle e^{3t}, e^{2t}, e^t \rangle] $
        \\
        $\frac{d}{dt} [\langle t^3e^t-te^{2t}, te^{3t}-t^2e^t, t^2e^{2t}-t^3e^{3t}\rangle]$
        \\
        $\frac{d}{dt}=\langle t^3e^t+3t^2e^t-2te^{2t}+e^{2t}, 3te^{3t}+e^{3t}-t^2e^t+ 2te^t, 2t^2e^{2t}+2te^{2t}-3t^3e^{3t}+3t^2e^{3t}\rangle$
        
        
    \end{enumerate}
    
    
    
    \item Derive the Chain Rule $\frac{d}{dt} {\bf r}(g(t))$ and use to find the derivative given:
    
   
    \begin{proof}
  $\frac{d}{dt} {\bf r}(g(t))$ Where ${\bf r}=\langle a(t),b(t),c(t)\rangle$
  \\
  $=\frac{d}{dt} \langle a(g(t)),b(g(t),c(g()t)\rangle$
  \\
    $= \langle g'(t)a'(g(t)),g'(t)b'(g(t),g'(t)c,(g(t)\rangle$
    \\
      $= g'(t)\langle a,(g(t)),b,(g(t),c,(g(t)\rangle$
      \\
      $\frac{d}{dt} {\bf r}(g(t))= g'(t) {\bf r}'(g(t)) $
    \end{proof}
    
    \begin{enumerate}
        \item ${\bf r}(t) = \langle t^2, 1-t \rangle$ and $g(t) = e^t$.
        \\
        Answer:  $ e^t \langle 2te^{t^2},e^t \rangle  $
        
        \item ${\bf r}(t) = \langle e^t, e^{2t}, 4 \rangle$ and $g(t) = 4t+9$.
        \\
        Answer: $4 \langle 4e^{4t+9}, 8e^{2(4t+9)},0 \rangle $
        
    \end{enumerate}

    \item Prove that if $||{\bf r}(t) || = c \in  \mathbb{R}$, then ${\bf r}(t)$ and ${\bf r}'(t)$ are orthogonal for all $t$.
    \\
    To shows two things are orthogonal we must show that $r \cdot r' =0$
    \\
    $||r(t)||=\sqrt{r(t) \cdot r(t)}=c$
    \\
    $||r(t)||^2=r(t) \cdot r(t) = c^2$
    \\
    $(r(t) \cdot r(t))' = (c^2)'$
    \\
    $r'(t) \cdot r(t)+r(t) \cdot r'(t) = 0 $
    \\
    $2(r(t) \cdot r(t)) = 0$
    \\
    $(r(t) \cdot r(t)) = 0$
    
    
    
    
    
    
    \item Evaluate the integrals, 
    \begin{enumerate}
    
    
        \item $\int_{-2}^2 (u^3 {\bf i} + u^5 {\bf j} \; du$
        \\
        Answer: $\langle \frac{1}{4}u^4|_{-2}^2, \frac{1}{6}u^6|_{-2}^2 \rangle$
        \\
        $\langle 0, 0$
        
        
        
        \item $\int_{0}^1 \langle 2t, 4t, -\cos 3t \rangle \; dt$
        \\
        Answer: $\langle t^2|_0^1, 2t^2|_0^1, -\frac{1}{3}\sin 3t |_0^1 \rangle$
        \\
        $\langle 1, 2, -0.047 \rangle$
        

        \item $\int \langle \frac{4}{t^2-t}, \frac{2t}{t^2+1}, \frac{4}{t^2+1} \rangle \; dt$
        \\
        Answer: \langle 4 ln (1-t)- ln(t)+c, ln(t^2+1)+c, 4 \tan^{-1}t+c \rangle



    \end{enumerate}
    
    
    
    
    \item Find the unit tangent vector to the curve ${\bf r}(t) = \langle t^2+1, t \rangle$.
    \\
    Answer:
    $T= \frac{r'}{||r'||} $
    \\
    $r'= \langle 2t, 1 \langle $
    \\
    $\frac{\langle 2t, 1 \rangle}{||2t,1||}=\frac{\langle 2t, 1 \rangle}{\sqrt{4t^2+1}} $
    
    \item Find the curvature of a straight line given by ${\bf r}(t) = \langle at+b, ct+d, e t+f \rangle$.
    \\
    Answer: $\mathcal{K}= \frac{r' \times r''}{|r'|^3} $
    \\
    $r'=\langle a,c,e \rangle $
    \\
    $r''=\langle 0,0,0 \rangle $
    \\
    $|r'|^3=|\sqrt{a^2+c^2+e^2}|^3 $
    \\
    $\frac{\langle a,c,e \rangle \times \langle 0,0,0 \rangle}{|\sqrt{a^2+c^2+e^2}|^3} $
    \\
    $=0$
    
 
 \item Find the curvature of a circle given by ${\bf r}(t) = \langle a \cos t + b, a \sin t +c \rangle$.
 \\
 Answer:  $\mathcal{K}= \frac{r' \times r''}{|r'|^3} $
 \\
 center $(b,c)$ with radius $a$
 \\
 $r'=\langle -a\sin t+ \cos t, a\cos t + \sin t \rangle  $
 \\
 $r''=\langle -a\cos t- \sin t, -a\sin t - \cos t \rangle $
 \\
 $|r'|^3= \sqrt{(-a\sin t + cos t)^2+(a \cos t + \sin t)}^3 $
 \\
 $\mathcal{K}= \frac{\langle -a\sin t+ \cos t, a\cos t + \sin t \rangle \times \langle -a\cos t- \sin t, -a\sin t - \cos t \rangle}{ \sqrt{(-a\sin t + cos t)^2+(a \cos t + \sin t)}^3} $
 \\
  $\mathcal{K}= \frac{1}{a}$
    
    
    

\end{enumerate}
\end{document}


