\documentclass[11pt]{article}

\usepackage{epsfig}
\usepackage{amsfonts}
\usepackage{amssymb}
\usepackage{amstext}
\usepackage{amscd}
\usepackage{amsmath}
\usepackage{xspace}
\usepackage{theorem}
\usepackage{float}
\usepackage[table]{xcolor}
\usepackage{color}
\usepackage{pgfplots}
\usepackage{pgf,tikz}
\usepackage{mathrsfs}
\usetikzlibrary{arrows}


\definecolor{stainlessSteel}{cmyk}{0,0,0.02,0.12}
%\usepackage{layout}% if you want to see the layout parameters
                     % and now use \layout command in the body

% This is the stuff for normal spacing
\makeatletter
 \setlength{\textwidth}{6.5in}
 \setlength{\oddsidemargin}{0in}
 \setlength{\evensidemargin}{0in}
 \setlength{\topmargin}{0.25in}
 \setlength{\textheight}{8.25in}
 \setlength{\headheight}{0pt}
 \setlength{\headsep}{0pt}
 \setlength{\marginparwidth}{59pt}

 \setlength{\parindent}{0pt}
 \setlength{\parskip}{5pt plus 1pt}
 \setlength{\theorempreskipamount}{5pt plus 1pt}
 \setlength{\theorempostskipamount}{0pt}
 \setlength{\abovedisplayskip}{8pt plus 3pt minus 6pt}
 \setlength{\intextsep}{15pt plus 3pt minus 6pt}

 \renewcommand{\section}{\@startsection{section}{1}{0mm}%
                                   {2ex plus -1ex minus -.2ex}%
                                   {1.3ex plus .2ex}%
                                   {\normalfont\Large\bfseries}}%
 \renewcommand{\subsection}{\@startsection{subsection}{2}{0mm}%
                                     {1ex plus -1ex minus -.2ex}%
                                     {1ex plus .2ex}%
                                     {\normalfont\large\bfseries}}%
 \renewcommand{\subsubsection}{\@startsection{subsubsection}{3}{0mm}%
                                     {1ex plus -1ex minus -.2ex}%
                                     {1ex plus .2ex}%
                                     {\normalfont\normalsize\bfseries}}
 \renewcommand\paragraph{\@startsection{paragraph}{4}{0mm}%
                                    {1ex \@plus1ex \@minus.2ex}%
                                    {-1em}%
                                    {\normalfont\normalsize\bfseries}}
 \renewcommand\subparagraph{\@startsection{subparagraph}{5}{\parindent}%
                                       {2.0ex \@plus1ex \@minus .2ex}%
                                       {-1em}%
                                      {\normalfont\normalsize\bfseries}}
\makeatother

\newcounter{thelecture}

\newenvironment{proof}{{\bf Proof:  }}{\hfill\rule{2mm}{2mm}}
\newenvironment{proofof}[1]{{\bf Proof of #1:  }}{\hfill\rule{2mm}{2mm}}
\newenvironment{proofofnobox}[1]{{\bf#1:  }}{}
\newenvironment{example}{{\bf Example:  }}{\hfill\rule{0mm}{0mm}} % change 2mm 2mm for square

%\renewcommand{\theequation}{\thesection.\arabic{equation}}
%\renewcommand{\thefigure}{\thesection.\arabic{figure}}

\newtheorem{fact}{Fact}
\newtheorem{lemma}[fact]{Lemma}
\newtheorem{theorem}[fact]{Theorem}
\newtheorem{definition}[fact]{Definition}
\newtheorem{corollary}[fact]{Corollary}
\newtheorem{proposition}[fact]{Proposition}
\newtheorem{claim}[fact]{Claim}
\newtheorem{exercise}[fact]{Exercise}

% math notation
\newcommand{\R}{\ensuremath{\mathbb R}}
\newcommand{\Z}{\ensuremath{\mathbb Z}}
\newcommand{\N}{\ensuremath{\mathbb N}}
\newcommand{\B}{\ensuremath{\mathbb B}}
\newcommand{\F}{\ensuremath{\mathcal F}}
\newcommand{\SymGrp}{\ensuremath{\mathfrak S}}
\newcommand{\prob}[1]{\ensuremath{\text{{\bf Pr}$\left[#1\right]$}}}
\newcommand{\expct}[1]{\ensuremath{\text{{\bf E}$\left[#1\right]$}}}
\newcommand{\size}[1]{\ensuremath{\left|#1\right|}}
\newcommand{\ceil}[1]{\ensuremath{\left\lceil#1\right\rceil}}
\newcommand{\floor}[1]{\ensuremath{\left\lfloor#1\right\rfloor}}
\newcommand{\ang}[1]{\ensuremath{\langle{#1}\rangle}}
\newcommand{\poly}{\operatorname{poly}}
\newcommand{\polylog}{\operatorname{polylog}}

% anupam's abbreviations
\newcommand{\e}{\epsilon}
\newcommand{\half}{\ensuremath{\frac{1}{2}}}
\newcommand{\junk}[1]{}
\newcommand{\sse}{\subseteq}
\newcommand{\union}{\cup}
\newcommand{\meet}{\wedge}
\newcommand{\dist}[1]{\|{#1}\|_{\text{dist}}}
\newcommand{\hooklongrightarrow}{\lhook\joinrel\longrightarrow}
\newcommand{\embeds}[1]{\;\lhook\joinrel\xrightarrow{#1}\;}
\newcommand{\mnote}[1]{\normalmarginpar \marginpar{\tiny #1}}

%%%%%%%%%%%%%%%%%%%%%%%%%%%%%%%%%%%%%%%%%%%%%%%%%%%%%%%%%%%%%%%%%%%%%%%%%%%
% Document begins here %%%%%%%%%%%%%%%%%%%%%%%%%%%%%%%%%%%%%%%%%%%%%%%%%%%%
%%%%%%%%%%%%%%%%%%%%%%%%%%%%%%%%%%%%%%%%%%%%%%%%%%%%%%%%%%%%%%%%%%%%%%%%%%%

\newcommand{\hwheadings}[3]{
{\bf Chelsea Blake Calculus 200 -  Fall 2015} \hfill {{\bf Problem Set #1}}\\
{{\bf } #2} \hfill {{\bf Due:} #3} \\
\rule[0.1in]{\textwidth}{0.025in}
%\thispagestyle{empty}
}

\begin{document}

\hwheadings{3}{Vectors and Vector Calculus }{Sept 23, 2015}


\begin{enumerate}

 \item Verify that $4(x \cdot y) = || x+y||^2 - ||x-y||^2$.
 \\
 Answer: We know that:
 \\
 $|| x+y||^2=(x+y) \cdot (x+y)$
 \\
 And:
 \\
 $ ||x-y||^2 = (x-y) \cdot (x-y)$
 \\
 Therefore
 \\
 $|| x+y||^2 - ||x-y||^2= (x+y) \cdot (x+y) -  (x-y) \cdot (x-y)$
 \\
 $x \cdot x + 2(x \cdot y) + y \cdot y - (x \cdot x - 2(x \cdot y) + y \cdot y)$
 \\
  $|| x+y||^2 - ||x-y||^2= 4(x \cdot y)$

 \item Prove the Triangle Inequality $||x+y|| \le ||x|| + ||y||$ 
 \\
 Answer:
 \\
 $||x+y||^2=(x+y) \cdot (x+y)$
 \\
  $||x+y||^2= x \cdot x + x \cdot y + y \cdot x + y \cdot y$
  \\
   $||x+y||^2= x \cdot x + 2(x \cdot y)+ y \cdot y$
   \\
    $||x+y||^2= ||x||^2 + 2(x \cdot y) + ||y||^2$
    \\
    $||x+y||^2 \le ||x||^2 + |2(x \cdot y)| + ||y||^2$
    \\
    $||x+y||^2 \le ||x||^2 + 2 ||x|| \cdot ||y|| + ||y||^2$
 \\
 $||x||^2 + 2 ||x|| \cdot ||y|| + ||y||^2 = (||x||+||y||)^2$
 \\
 $||x+y||^2 \le (||x||+||y||)^2$
 \\
  $||x+y|| \le (||x||+||y||)$
 
 
 \item Compute the dot product ${\bf a} \cdot {\bf b}$ for ${\bf a} = \langle 1,2,3 \rangle$ and ${\bf b} = \langle 5,-3,4 \rangle$.
 \\
 Answer:
 \\
 $(1 \times 5)+ (2 \times -3)+ (3 \times 4)$
 \\
 $5-6+12=11$
 
 
 \item Find the angle between the vectors $\langle 2,1,-3 \rangle$ and $ \langle 1,5,6 \rangle$.
 \\
 Answer:
 \\
 $x \cdot y = ||x|| \times ||y|| \times \cos \theta$
 \\
 $\langle 2,1,-3 \rangle \cdot \langle 1,5,6 \rangle = \sqrt{2^2+1^2+3^2} \times \sqrt{1^2+5^2+6^2} \times \cos \theta $
 \\
 $2+5-18= \sqrt{14} \times \sqrt{62} \times \cos \theta $
 \\
 $-11=\sqrt{868} \cos \theta$
 \\
 $\frac{-11}{\sqrt{868}}= \cos \theta $
 \\
 $\theta= \cos^{-1}\frac{-11}{\sqrt{868}} $
 \\
 $\theta= 1.9$ rad or $\theta = 111.9^{o} $
 
 
 
 
 \item Find the projection of ${\bf a} = \langle 2,3 \rangle$ onto ${\bf b} = \langle -1,5 \rangle$.
 \\
 Answer:
 \\
 $\frac{a \cdot b}{|b|^2} \times b $
 \\
 $\frac{(2)(-1)+(3)(5)}{(-1)^2+(5)^2} \times (-1,5) $
 \\
 $\frac{1}{2} \times (-1,5) $
 \\
 $(\frac{-1}{2}, \frac{5}{2}) $
 
 
 \item Compute $\langle 1, 2, 3 \rangle \times \langle 4, 5, 6 \rangle$.
 \\
 Answer:
 \\
 $\langle 1, 2, 3 \rangle \times \langle 4, 5, 6 \rangle= \langle 2 \times 6 - 3 \times 5, 3 \times 4 - 1\times 6, 1 \times 5 - 2 \times 4 \rangle $
 \\
 $=\langle 12-15, 12-6, 5 - 8 \rangle$
 \\
 $=\langle -3, 6, -3 \rangle$
 
 
 
 
 \item Use the Cauchy-Schwartz inequality in $n$ dimensions to show that $\sum \limits_{k=1}^n |a_k| \le \sqrt{n} \left(\sum \limits_{k=1}^n a_k^2 \right)^{1/2}$.  Then use this result to prove that if $p_1, p_2, \dots , p_n$ are non-negative numbers that sum to $1$, then $\sum \limits_{k=1}^n p_k^2 \ge \frac{1}{n}$.
 \\
 Answer:
 \\
 Cauchy-Shwartz inequality: $|x \cdot y| \le ||x|| ||y|| $
\\
$||x||= \sqrt{\sum \limits_{k=1}^{n} x_k^2}$
\\
$||y||= \sqrt{\sum \limits_{k=1}^{n} y_k^2}$
\\
Let $x = \langle 1,1,1,1,1.....,1 \rangle$
\\
and $ y= \langle |a_1|,|a_2|,.....,|a_n| \rangle$
 \\
 $|x \cdot y|= x \cdot y$
 \\
 $= \sum \limits_{k=1}^{n} |a_k| \le \sqrt{n} (\sqrt{\sum \limits_{k=1}^{n} a_k^2})$
 \\
 p_i=|a_i|
 \\
 $\sum \limits_{i=1}^{n} |p_i| \le \sqrt{n} (\sqrt{\sum \limits_{i=1}^{n} p_i^2})$
 \\
 $\frac{1}{n} \le \sum \limits_{i=1}^{n} p_i^2}
 
 
 
 
 
 \item Prove that for any vectors ${\bf a}$ and ${\bf b}$ in $\R^3$, ${\bf a} \times {\bf b}$ is orthogonal to both ${\bf a}$ and ${\bf b}$.
 \\
 Answer:
 \\
 Let $a=\langle a_1,a_2,a_3 \rangle$ and $b= \langle b_1, b_2, b_3 \rangle$
 \\
 $a \cdot (a \times b)=$
 \\
 $ a \cdot \lanlge a_2 b_3 - a_3 b_2, a_3 b_1 - a_1 b_3, (a_1 b_2 - a_2 b_1 \rangle$
 \\
 $a_1(a_2 b_3 - a_3 b_2) + a_2 (a_3 b_1 - a_1 b_3) + a_3 (a_1 b_2 - a_2 b_1)$

 \\
$a_1 a_2 b_3 - a_1 a_3 b_2 + a_2 a_3 b_1 - a_1 a_2 b_3 + a_1 a_3 b_2 - a_2 a_3 b_1= 0$
\\
The same can be seen for  $b \cdot (a \times b)=$
\\
$ b \cdot \lanlge a_2 b_3 - a_3 b_2, a_3 b_1 - a_1 b_3, (a_1 b_2 - a_2 b_1 \rangle$
\\
 $b_1(a_2 b_3 - a_3 b_2) + b_2 (a_3 b_1 - a_1 b_3) + b_3 (a_1 b_2 - a_2 b_1)$
 \\
 $b_1 a_2 b_3 - b_1 a_3 b_2 + b_2 a_3 b_1 - a_1 b_2 b_3 + a_1 b_3 b_2 - a_2 b_3 b_1= 0$
 \\
 By definition, if the dot product is equal to zero, the vectors are orthogonal.

 
 \item Prove that for nonzero vectors ${\bf a}$ and ${\bf b}$ in $\R^3$, if $0 \le \theta \le \pi$ is the angle between ${\bf a}$ and ${\bf b}$, then $|| {\bf a} \times {\bf b}|| = || {\bf a}|| \; || {\bf b}|| \; \sin \theta$.
 \\
 Answer:
 \\
 We know that $|| a \times b||^2= ||a||^2 ||b||^2 - (a \cdot b)^2$
 \\
 we also know that $(a \cdot b) = ||a|| ||b|| \cos \theta $
 \\
 $|| a \times b||^2= ||a||^2 ||b||^2 - (||a|| ||b|| \cos \theta)^2$
 \\
 $=||a||^2 ||b||^2 - ||a||^2 ||b||^2 \cos^2 \theta $
 \\
 $=||a||^2 ||b||^2 - (1 - \cos^2 \theta) $
 \\
 $|| a \times b||^2=||a||^2 ||b||^2 - \sin^2 \theta $
 \\
 $|| a \times b||=||a|| ||b|| - \sin \theta $






 \item Derive a formula for area of  a parallelogram then use it to find the area formed from the vectors $\langle 1, 2, 3 \rangle$ and $\langle 4, 5, 6 \rangle$.
 \\
 Answer:
 \\
\definecolor{xdxdff}{rgb}{0.49019607843137253,0.49019607843137253,1.}
\definecolor{qqqqff}{rgb}{0.,0.,1.}
\begin{tikzpicture}[line cap=round,line join=round,>=triangle 45,x=1.0cm,y=1.0cm, scale=0.75]
\clip(0.54,2.58) rectangle (9.48,8.46);
\draw (3.58,7.16)-- (2.72,5.28);
\draw (2.72,5.28)-- (5.86,5.22);
\draw (5.86,5.22)-- (6.46,7.14);
\draw (6.46,7.14)-- (3.58,7.16);
\draw (3.58,7.16)-- (3.5000973314948496,5.265093681563793);
\draw (2.48,6.74) node[anchor=north west] {$a$};
\draw (3.92,5.18) node[anchor=north west] {$||b||$};
\draw (3.,5.98) node[anchor=north west] {$\theta$};
\draw (3.66,6.54) node[anchor=north west] {$h=||a|| \sin \theta$};
\draw [->] (4.74,4.72) -- (5.76,4.72);
\draw [->] (3.8,4.74) -- (2.72,4.74);
\begin{scriptsize}
\draw [fill=qqqqff] (3.58,7.16) circle (1.5pt);
\draw [fill=qqqqff] (2.72,5.28) circle (1.5pt);
\draw [fill=qqqqff] (5.86,5.22) circle (1.5pt);
\draw [fill=qqqqff] (6.46,7.14) circle (1.5pt);
\draw [fill=xdxdff] (3.5000973314948496,5.265093681563793) circle (1.5pt);
\end{scriptsize}
\end{tikzpicture}
\end{tikzpicture}
\\
 We know that the area for a parallelogram is the base times the height. Here we let vector $b$ be the base of the parallelogram. Vector $a$ is the diagonal line connecting the two bases. to find the height we draw a vertacle line from the uppermost point of $a$ and straight down. This line represents the height. To find this line we take the $\sin$ of the angle between $a$ and $b$ and multiply it by $a$
 \\
 $a= ||a|| \times ||b|| \sin \theta $
 \\
 This is also equal to:
 \\
 $||a \times b||$
 \\
$a = ||\langle 2(6)-3(5), 3(4) - 1(6), 1(5) - 2(4) \rangle||$
\\
$a  = ||\langle 12 - 15, 12 - 6, 5 - 8 \rangle||$ 
\\
$a = ||\langle -3, 6, -3 \rangle||$ 
\\
$a = \sqrt{(-3)^2+(6)^2+(-3)^2}$  
\\
 $a = \sqrt{54}$
 


 \item 11. Find the distance from the point $(1, 2, 1)$ to the line through the points $(2, 1, -3)$ and $(2, -1, 3)$. 
 \\
 Answer:
 \\
 The line  through the points $(2, 1, -3)$ and $(2, -1, 3)$ gives you $\langle 0, -2, 6 \rangle$
 \\
 We then use the equation $\sqrt{(x_2-x_1)^2+(y_2-y_1)^2+(z_2-z_1)^2}$
 \\
 $\sqrt{(0-1)^2+(-2-2)^2+(6-3)^2}$
 \\
 $\sqrt{26}$
 
 \item 12. Given the data set \textit{Voting}, determine how similar the voting records of (then) Senators Barack Obama and Hillary Clinton.  Describe an algorithm that would rank the similarity in voting for Barack Obama. 
 \\
 Answer:
 \\
 Similarity = $ \cos \theta= \frac{a \cdot b}{||a|| ||b||}$
 \\
 a (obama)= $\langle 1, -1, 1, 1, 1, -1, -1, -1, 1, 1, 1, 1, 1, 1, -1, 1, 1, 1, 1, 1, 1, 1, 1, 1, -1, 1, -1, -1, 1, 1, 1, 1, 1, 1, 1, 1, 1, -1, 1, 1, 1, 1, -1, 1, 1, -1 \rangle$
 \\
 b (clinton)= $ \langle -1, 1, 1, 1, 0, 0, -1, 1, 1, 1, 1, 1, 1, 1, -1, 1, 1, 1, 1, 1, 1, 1, 1, 1, -1, 1, -1, 1, 1, 1, 1, 1, 1, 1, 1, 1, 1, -1, 1, 1, 1, 1, -1, 1, 1, 1 \rangle  $
 \\
 $a \cdot b = -1-1+1+1+0+0+1-1+1+1+1+1+1+1-1+1-1+1+1+1+1+1+1+1-1+1+1-1-1+1+1+1+1+1+1+1-1+1-1+1+1-1+1-1+1+1+1-1 $
 \\
  $a \cdot b = 34 $
  \\
  $||a||= \sqrt(46)= 6.78$
  \\
  $||b||= \sqrt{44}=6.633$
  \\
   $ \cos \theta=  \frac{34}{\sqrt{46}\sqrt{44}}$
   \\
   $ \cos \theta= 0.755$
   \\
   $75 \%$ of the time Obama and Clinton are in agreement 
   \\
   algerithm: find the dot product of each person and oboma and then rank them from lowest to highest.
   
   
 
 
 
 \item 13. Derive the formula for the volume of a parallelepiped using the Cross product then use it to find the volume formed by the vectors $\langle 1, 2, 3 \rangle, \langle 4, 5, 6 \rangle$ and $\langle 7, 8, 0 \rangle$.
 \\
 Answer:
 \\
 Volume= area of base $\times$ height
 \\
 We found in problem 10 that the area of the base is $||a \times b||$
 \\
 The height is vector $c$ at the normal direction of $a \times b$ therefore, the height is $||c|| \times |\cos \theta|$ where $\theta$ is the angle between the vectore $a \times b$ and $c$
 Finally we get:
 \\
 $a=||a \times b|| \times ||c|| \times |\cos \theta|$
 \\
 Which is also equal to 
 \\
 $|(a \times b) \cdot c|$ 
 \\
 For finding the volumed formed by the vectors $\langle 1, 2, 3 \rangle, \langle 4, 5, 6 \rangle$ and $\langle 7, 8, 0 \rangle$ we do:
 \\
 $|(\langle 1, 2, 3 \rangle \times \langle 4, 5, 6 \rangle) \cdot \langle 7, 8, 0 \rangle|$
 \\
 $\langle (2)(6)-(3)(5), (3)(4)-(1)(6), (1)(5)-(2)(4)\rangle$
 \\
 $\langle 12-15, 12-6, 5-8 \rangle$
 \\
 $\langle -3,6,-3 \rangle$
 \\
 $|\langle -3,6,-3 \rangle \cdot \langle 7, 8, 0 \rangle|$
 \\
 $(3)(7)+(6)(8)+(-3)(0) $
 \\
 $-21+48+0 $
 \\
 $|27|$
 \\
 $27$
 
 
 \item 14. Determine parametric equations for a line in $\R^3$ through a point $P_1$ parallel to a vector ${\bf a}$.  Hint: Given a point $P_1=(x_1, y_1, z_1)$ on the line, use an arbitrary point on the line to find the vector parallel to ${\bf a}$. 
 \\
 Answer:
 \\
 $x_1= X_o + ta$
 \\
 $y_1=y_o+ tb$
 \\
 $z_1=z_o+tc$
 
 \item We describe a plane using a point that lies in the plane and a normal vector.  Let $P_1=(x_1, y_1, z_1)$ and any other arbitrary point $P=(x,y,z)$ that lies in the plane and a vector ${\bf a} = \langle a, b, c \rangle$ normal to the plane.  Derive the equation of the plane and use it to find the equation of the plane containing the point $(1, 2, 3)$ with normal vector $\langle 4, 5, 6 \rangle$.   
 \\
 Answer:

 \\
$a(x-x_o)+b(y-y_o)+c(z-z_o)=0$
\\
$4(x-1)+5(y-2)+6(z-3)=0$
\\
$4x-4+5y-10+6z-18=0 $
\\
$4x+5y+6z=32 $







\end{enumerate}
\end{document}

