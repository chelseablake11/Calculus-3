\documentclass[11pt]{article}

\usepackage{epsfig}
\usepackage{amsfonts}
\usepackage{amssymb}
\usepackage{amstext}
\usepackage{amscd}
\usepackage{amsmath}
\usepackage{xspace}
\usepackage{theorem}
\usepackage{float}
\usepackage[table]{xcolor}
\usepackage{color}
\usepackage{pgf,tikz}
\usepackage{mathrsfs}

\definecolor{stainlessSteel}{cmyk}{0,0,0.02,0.12}
%\usepackage{layout}% if you want to see the layout parameters
                     % and now use \layout command in the body

% This is the stuff for normal spacing
\makeatletter
 \setlength{\textwidth}{6.5in}
 \setlength{\oddsidemargin}{0in}
 \setlength{\evensidemargin}{0in}
 \setlength{\topmargin}{0.25in}
 \setlength{\textheight}{8.25in}
 \setlength{\headheight}{0pt}
 \setlength{\headsep}{0pt}
 \setlength{\marginparwidth}{59pt}

 \setlength{\parindent}{0pt}
 \setlength{\parskip}{5pt plus 1pt}
 \setlength{\theorempreskipamount}{5pt plus 1pt}
 \setlength{\theorempostskipamount}{0pt}
 \setlength{\abovedisplayskip}{8pt plus 3pt minus 6pt}
 \setlength{\intextsep}{15pt plus 3pt minus 6pt}

 \renewcommand{\section}{\@startsection{section}{1}{0mm}%
                                   {2ex plus -1ex minus -.2ex}%
                                   {1.3ex plus .2ex}%
                                   {\normalfont\Large\bfseries}}%
 \renewcommand{\subsection}{\@startsection{subsection}{2}{0mm}%
                                     {1ex plus -1ex minus -.2ex}%
                                     {1ex plus .2ex}%
                                     {\normalfont\large\bfseries}}%
 \renewcommand{\subsubsection}{\@startsection{subsubsection}{3}{0mm}%
                                     {1ex plus -1ex minus -.2ex}%
                                     {1ex plus .2ex}%
                                     {\normalfont\normalsize\bfseries}}
 \renewcommand\paragraph{\@startsection{paragraph}{4}{0mm}%
                                    {1ex \@plus1ex \@minus.2ex}%
                                    {-1em}%
                                    {\normalfont\normalsize\bfseries}}
 \renewcommand\subparagraph{\@startsection{subparagraph}{5}{\parindent}%
                                       {2.0ex \@plus1ex \@minus .2ex}%
                                       {-1em}%
                                      {\normalfont\normalsize\bfseries}}
\makeatother

\newcounter{thelecture}

\newenvironment{proof}{{\bf Proof:  }}{\hfill\rule{2mm}{2mm}}
\newenvironment{proofof}[1]{{\bf Proof of #1:  }}{\hfill\rule{2mm}{2mm}}
\newenvironment{proofofnobox}[1]{{\bf#1:  }}{}
\newenvironment{example}{{\bf Example:  }}{\hfill\rule{0mm}{0mm}} % change 2mm 2mm for square

%\renewcommand{\theequation}{\thesection.\arabic{equation}}
%\renewcommand{\thefigure}{\thesection.\arabic{figure}}

\newtheorem{fact}{Fact}
\newtheorem{lemma}[fact]{Lemma}
\newtheorem{theorem}[fact]{Theorem}
\newtheorem{definition}[fact]{Definition}
\newtheorem{corollary}[fact]{Corollary}
\newtheorem{proposition}[fact]{Proposition}
\newtheorem{claim}[fact]{Claim}
\newtheorem{exercise}[fact]{Exercise}

% math notation
\newcommand{\R}{\ensuremath{\mathbb R}}
\newcommand{\Z}{\ensuremath{\mathbb Z}}
\newcommand{\N}{\ensuremath{\mathbb N}}
\newcommand{\B}{\ensuremath{\mathbb B}}
\newcommand{\F}{\ensuremath{\mathcal F}}
\newcommand{\SymGrp}{\ensuremath{\mathfrak S}}
\newcommand{\prob}[1]{\ensuremath{\text{{\bf Pr}$\left[#1\right]$}}}
\newcommand{\expct}[1]{\ensuremath{\text{{\bf E}$\left[#1\right]$}}}
\newcommand{\size}[1]{\ensuremath{\left|#1\right|}}
\newcommand{\ceil}[1]{\ensuremath{\left\lceil#1\right\rceil}}
\newcommand{\floor}[1]{\ensuremath{\left\lfloor#1\right\rfloor}}
\newcommand{\ang}[1]{\ensuremath{\langle{#1}\rangle}}
\newcommand{\poly}{\operatorname{poly}}
\newcommand{\polylog}{\operatorname{polylog}}

% anupam's abbreviations
\newcommand{\e}{\epsilon}
\newcommand{\half}{\ensuremath{\frac{1}{2}}}
\newcommand{\junk}[1]{}
\newcommand{\sse}{\subseteq}
\newcommand{\union}{\cup}
\newcommand{\meet}{\wedge}
\newcommand{\dist}[1]{\|{#1}\|_{\text{dist}}}
\newcommand{\hooklongrightarrow}{\lhook\joinrel\longrightarrow}
\newcommand{\embeds}[1]{\;\lhook\joinrel\xrightarrow{#1}\;}
\newcommand{\mnote}[1]{\normalmarginpar \marginpar{\tiny #1}}

%%%%%%%%%%%%%%%%%%%%%%%%%%%%%%%%%%%%%%%%%%%%%%%%%%%%%%%%%%%%%%%%%%%%%%%%%%%
% Document begins here %%%%%%%%%%%%%%%%%%%%%%%%%%%%%%%%%%%%%%%%%%%%%%%%%%%%
%%%%%%%%%%%%%%%%%%%%%%%%%%%%%%%%%%%%%%%%%%%%%%%%%%%%%%%%%%%%%%%%%%%%%%%%%%%

\newcommand{\hwheadings}[3]{
{\bf Calculus 200 -  Fall 2015} \hfill {{\bf Problem Set #1}}\\
{{\bf } #2} \hfill {{\bf Due:} #3} \\
\rule[0.1in]{\textwidth}{0.025in}
%\thispagestyle{empty}
}

\begin{document}

\hwheadings{2}{Chelsea Blake: Calculus of Parametric equation }{Sept 16, 2015}


\begin{enumerate}

 \item Find the parametric equations describing the given curve:
 
 
 
  \begin{enumerate}

    \item Line segment from $(0, 1)$ to $(3,4)$
    \\
    Answer: $x=t$
    \\
    $Y=t+1$
    \\
    where $0 \le t \le 3$
    
    \item Line segment from $(3, 1)$ to $(1,3)$
    \\
    Answer: $x=3-t$
    \\
    $y=t+1$
    \\
    where $0 \le t \le 2$
  
    \item Portion of the parabola $y=x^2+1$ from $(1, 2)$ to $(2,5)$
    \\
    Answer: $x=t$
    \\
    $y=t^2+1$
    \\
    where $1 \le t \le 2$
    
    
   \item Circle of radius 3 centered at $(2, 1)$, counterclockwise
\end{enumerate}   
\\
Answer: $x=3 \cos t +2$
\\
$y=3 \sin t +1$
\\
Where $0 \le t \le 2 \pi$



    
 \item Find parametric equations for the path of a projectile launched from height $h=16$ feet with initial speed $v = 12$ ft/s at angle $\theta = 6^\circ$ from the horizontal.   
    
    Answer: $h=16 V=12 \theta = 6$
    \\
    $y=-16t^2+1.25t+16$
    \\
    $y=-16t^2+12 \sin (6t) +16$
    \\
    $x=11.93t$
    \\
    $x=12 \cos (6t)$
    
    
    
    
    



     \item Find the slope of the tangent line to the curves $x=t^2-2$ and $y=t^3-t$ at
 \begin{enumerate}

   \item $t=-1$
   \\
   Answer: slope= $-1$
   
     \item $t=1$
     \\
     Answer: slope=$1$
     
   \item $(-2,0)$
   \\
   Answer: slope= $ \lim{t \to 0} \frac{3t^2-1}{2t}= - \infty$

\end{enumerate}    
    

    
\item Identify all points at which the curve $x=\cos 2t$ and $y=\sin 4t$ has a horizontal and vertical tangent.  
\\
Answer: $\frac{\frac{dy}{dt}}{\frac{dx}{dt}}$
\\
$y'=4 \cos (4t)$
\\
$x'=-2\sin (2t)$
\\
horizontal: $\frac{4 \cos (4t)}{-2 \sin (2t)}=0$
\\
$\cos (4t) =0$
\\
$t= \{ \frac{\pi}{8}, \frac{3\pi}{8}, \frac{5 \pi}{8}, \frac{7 \pi}{8} \}$
\\
$(x,y)=(\frac{\sqrt{2}}{2},1),(\frac{-\sqrt{2}}{2}, 1),(\frac{\sqrt{2}}{2},1), (\frac{-\sqrt{2}}{2}, 1)  $

\\
Vertical: find were graph is undefined, i.e. where the denominator equals 0
\\
$-2 \sin (2t)=0$
\\
$t= \{\frac{\pi}{2}, \pi \}$
\\
$(x,y)=(-1,1),(1,0)
    
    
 \item Find the object's speed at the given time and describe its motion where $x=2 \cos t$ and $y=3 \sin t$ at $t = \frac{\pi}{2}$.   
    \\
    Answer: $x'=-2\sin t$
    \\
    $y'=3 \cos t$
    \\
    $\sqrt{(-2\sin(\frac{\pi}{2}))^2+(3\cos(\frac{\pi}{2}))^2}=\sqrt{4}=2$








\item  Find the area enclosed by the curve $x=t^3 - 4t$ and $y=t^2-3$ where $-2 \le t \le 2$.
\\
Answer: $- \int_{\alpha}^{\beta} y(t) x'(t) dt$
\\
$- \int_{-2}^{2} (t^2-3)(3t^2-4) dt$
\\
$- \int_{-2}^{2} 3t^4-13t^2+12 dt$
\\
$[\frac{-3}{5}t^5+\frac{13}{3}t^3-12t]_{-2}^{2}$
\\
$= -\frac{256}{15}=-17.1$



\item Find the arc length of the curve $x=t \cos t$ and $y = t \sin t$ for $-1 \le t \le 1$.  
\\
Answer: $\int_{-1}^{1} \sqrt{(-t\sin(t)+\cos(t))^2+(t\cos (t) + sin (t)^2}$
\\
$\sqrt{2}+\sin^-1(1)=2.29$


\item Sketch the location of the points $(1, 1, 0), (2, 3, -1)$, and $(-1, 2, 3)$ on a single set of axes.
\\
*DO NOT DO*


\item Find the lengths of the sides of the triangle with vertices $(1, 0, 1), (2, 2, -1)$, and $(-3, 2, -2)$.
\\
Answer:
$\sqrt{(1-2)^2+(0-2)^2+(1-(-1))^2}=3$
\\
$\sqrt{(1-(-3))^2+(0-2)^2+(1-(-2))^2}=5.39$
\\
$\sqrt{(2-(-3))^2+(2-2)^2+(1-2)^2}=5.099$



\item Find an equation of the sphere with center at $(1, 1, 1)$ and radius $2$. 
\\
Answer: $(x-1)^2+(y-1)^2+(z-1)^2$


\item Describe geometrically the set of points $(x, y, z)$ that satisfy $z = 4$.
\\
Answer: geometrically, you would have a plane that sits up four units on the z axes.
\item Describe geometrically the set of points $(x, y, z)$ that satisfy $y = -3$.
\\
Answer: geometrically, you would have a plane that sits up -3 units on the y axes.
\item Describe geometrically the set of points $(x, y, z)$ that satisfy $x + y = 2$.
\\
Answer: geometrically, you would have a diagonal plane that cuts vertically in between the x and y axes. This plane hits the two axes at x=2 and y=2
\item The equation $x + y + z = 1$ describes some collection of points in $\mathbb{R}^3$. Describe and sketch the points that satisfy $x + y + z = 1$ and are in the $x-y$ plane, in the $x-z$ plane, and in the $y-z$ plane.
\\
Answer: Geometrically, you would have a plane along the points $(1,0,0),(0,1,0),(0,0,1)$




\end{enumerate}
\end{document}
